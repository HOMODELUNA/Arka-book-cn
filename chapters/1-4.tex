\chapter[时态]{时态}
%\chapter[短标题显示在页面]{长标题显示在目录}

% {\small\textit{
% \quad Sors de l'enfance, ami réveille toi!\newline
% \quad \quad —Jean Jacques Rousseau.}
% \footnote{Ekster el infaneco,amiko veku!}
% \\}

\begin{figure}[H]
\includegraphics[width=1\textwidth]{ARKA/tier2.jpg}
\end{figure}

Lein: ``我们来海边啦!好-大-的-海-啊!嘿,紫苑,你在干什么?''

紫苑: ``唔...你的腿好细啊.如果我是你的话...''

Lein: ``额...''

紫苑: ``诶~''xion in lein``哦.上次我们学了''紫苑看到了lein.``你也说Arka是SVO型语言''

Lein: ``对对对,形容词在名词后面,记得挺牢的.''


\emoji{x_knoos}
那么,你怎么说``紫苑看到了Lein?''


\emoji{l_sena}
过去时啊,在汉语里,在动词后面加上``了'',在Arka则是``-at''.


\emoji{x_lo}
So,我试着想,``紫苑看到了Lein''就是``xion inat lein,''.

Arka 没有``saw''之类要特殊记的变化,这倒挺简单.没有啥例外吧?


\emoji{l_lo}
我想想...哦哦,动词结尾是元音的话,就不是加``at''而是``-t''.

``Xa''是``在某处'',结尾是元音,所以加 ``-t''.``Xat''就是 ``曾经在某处.''


\emoji{x_loki}
避免元音冲突?

还有``-at''同类型的词缀吗?


\emoji{l_niit}
要表示进行的体
\footnote{译者注:英语原文用的是aspect,日语原文用的是``過去形'',``進行形'',``完了形'',
既然日语原文没有明确是时态(tense)还是体态(aspect),暂且以英语原文为准,说这个词缀表示动词的体态.}
的话,
就在动词后面加``-or''.Axt``是''写``, ''axtor``就是 ''正在写."

如果动词以元音结尾就不加``-or'',而是``-r.'' ,``Ena''是``哭'',那么``enar''就是``正在哭.''


\emoji{x_asex}
在进行体的表述上,Arka和汉语真是不一样.

``-Or''表示进行体.我觉得你下次该说完成体了.


\emoji{l_diina}
猜对了!完成体是在动词后面加``-ik''.

``Axtik''是``已经写了.''

要是动词以元音结尾,就只用加后缀``-k.''


\emoji{x_niit}
``-At''是过去时.``-Or''是进行体.``-ik''是完成体.

等会,将来的事件怎么办?


\emoji{l_uni}
我们不用后缀来表示将来时.

下次我再说吧

\begin{figure}[H]
\includegraphics[width=1\textwidth]{ARKA/tier.jpg}
\end{figure}

\emoji{x_knoos}
顺便问一下,Lein,我们现在看的是那片海啊?


\emoji{l_iks}
阿这,我们,在,在,在首都Arna南边,边,边的,的Kateej...(´·ω·`)


\emoji{x_pilp}
其实就是静冈县 :-7


\emoji{l_vexl}
诶呀,这是善意的谎言啦\~{}


%``([a-z]+)''
%``$1''



\chapter[关系词]{关系词}
%\chapter[短标题显示在页面]{长标题显示在目录}

\emoji{x_niit} 
嘿 Lein,我们现在学了很多语法了,还有多少?


\emoji{l_deyu} 
语法的话这就是最后一节了.

这节是关于关系代词的.


\emoji{x_lo} 
我学英语的关系代词是总是非常懵.

我们看这一个句子 "I know a girl who is cute."
如果先行词不是"girl"而是"book"的话,"who"就会变成"which."
在"I know a girl whom he likes"里又是"whom",因为这里的女生是宾语.


\emoji{l_asex_kal} 
对啊

准确的来说,Arka并没有关系代词."Who","which"和"whom"在Arka里都是"le",但"le"是连词.

试着把"I know a girl who likes cats"翻译成Arka?

Arka和English有一样的句子结构,所以我就给你一点提示.

know = ser

a girl = fian

likes = siina


\emoji{x_demo} 

哦,有点难啊.

句子结构是一样的,"who"在Arka是"le",so--
"an ser fian le siina ket"?


\emoji{l_ket} 
干得漂亮!

那么,"I know a girl whom he likes"又如何呢?


\emoji{x_pels} 
"He"是"lu"或"la."我们现在用"lu".

用"who"还是"whom"都可以,因为Arka只有"le"--
"an ser fian le lu siina,"对吧?


\emoji{l_asex_kal} 
Yeah.即使是"book"而不是"girl",你也可以用"le".我是说Arka不区分"who"或"which."

另外, 如果"le lu siina"变成了"le alis siina,"你就应该说"l'alis siina."


\emoji{x_lo} 
在原有开头的词前面,"le"就变成了"l".


\emoji{l_diina} 
对, 对.

你以后就能用"le"造复杂的句子,所以习惯就好.


\emoji{x_nal} 
OK,这样语法就告一段落了吧.


\emoji{l_niit} 
上了我这么长的课程你已经累了吧.

我们下次就远离语法了.


%``([a-z]+)''
%``$1''



\chapter[Arka与其他的人造语言有什么不同]{Arka与其他的人造语言有什么不同}

\subsubsection{1:拥有先验的人造语言}

新生的Arka是以古Arka和制Arka为基础而形成的。
这些都是先验的。简单来说,就是不抄袭英语等自然语言的意思。
但是,完全排他性地排斥自然语言并实现先验是在中后期Arka以后。

\subsubsection{2:拥有先验的人工文化}

Arka拥有独特的文化设定。
avom(狼)是军队和士兵的象征,比起凶猛、残忍的负面形象,更具有勇猛、果敢的形象。
这是没有文化设定的语言无法决定的。没有文化设定就无法定义狼的形象。
即使没有文化,也可以定义狼作为生物物种的名字,但不能设定象征和形象。

\subsubsection{3:具有先验的人工风土}

Arka拥有独特的人工风土Atlas。Atlas相当于虚构的地球。
主要使用Arka的国家阿尔巴莎德相当于地球上所说的南法。
因为小麦是主食,所以pof(面包)和koka(切片)等作为单纯词分化出来。相反,大米和稻子是同一个单词past。
如果没有风土设定,就无法定义如何划分世界。

另外,文化Antis和风土Atlas相加的就是人工世界卡勒底亚。

把人工语言Arka和人工世界Kaldia加在一起,就叫做幻奏。

语言、文化、风土都是先验的,而且做到这种程度的只有幻奏

语言、文化、风土都是先验的,而且被塑造到这种程度的,除了幻奏以外别无他法。(2010年)

另外,实际上在网络和现实中使用的先验语言这一点也非常有特点。(同年)

(一般来说,后验式人造语言更容易应用。先验的实用难度很高)

世界语等,是有名的后验语,所谓的模仿自然语言的语言。
与先验比较,后验语不费功夫,所以马上就能做出来。

\subsubsection{总结}

正如多次重复先验的那样,独创性和创造性是Arka的特点。

原创词汇、独特的虚构文化等,把重点放在编织“不依赖任何东西的纯粹的空想世界”上。特别强调艺术要素。
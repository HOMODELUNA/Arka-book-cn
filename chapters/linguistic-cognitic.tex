\chapter[认知语言学层面的Arka]{认知语言学层面的Arka}
%\chapter[短标题显示在页面]{长标题显示在目录}
%Arka的\hypertarget{appendix-pronouns}{位相}

\section{客观把握与主观把握}
在认知语言学中,认知主体可以定义为对事态的把握.认知的模式可以分为两大类.

一种是从事态的外部把握事态的样式,相当于池上(2002,2005)
\footnote{池上嘉彦(2004)「言語における〈主观性〉と〈主观性〉の言語的指標 (1)」『認知言語学論考』No.3 pp1-49 ひつじ書房
――(2005)「言語における〈主观性〉と〈主观性〉の言語的指標 (2)」『認知言語学論考』No.4 pp1-60 ひつじ書房}所说的客观把握.

一种是从事态内部把握事态的方式,同样相当于主观把握.

事实上同一概念,语言学家用各自的术语来表示.
例如,Langacker(1985)将客观把握称为最佳视点阵列(optimal viewing arrangement),中村(2004)设为D模式.
同样是主观把握,Langacker用自我中心的视点排列(egocentric viewing arrangement),中村用I模式.
本文以最容易从字面理解意思为由,模仿池上的术语.

在语言学上并不矛盾的人工语言的制作方法中,"做"型语言(する语言)相当于客观的把握,"成为"型语言(なる语言)相当于主观的把握.
前文中所涉及的特征数量很少,如做·成为,物·事、have·be等,主要是用"做"和"成为"的对比来说明的.
但是,由于本文所涉及的特征数量太多,所以用する语言、なる语言反而会变得晦涩难懂.因此,从这里开始,要把术语统一为客观把握和主观把握.

当然,各个术语的意思各不相同.
例如中村(2004p40)将I - mode称为``西田的认识论'',将D - mode称为``笛卡尔的认识论'',从哲学角度进行了分析.
在这个意义上,术语的语感各不相同,本文为了方便,统一使用池上的术语.

\section{Arka掌握事态的方法}

在前面的文章中,我们已经明确Arka具有客观把握和主观把握两方面的特征.
森山(2009)提出:一般来说,语言的认知主体进行客观把握,很少夹杂主观把握.

将外界事态语言化有``客观把握''和``主观把握''两种把握方法,根据语言的不同,选择哪一种优先,左右着每种语言所具有的整体类型论特征.
但客观把握和主观把握不是规律,而是倾向,所以即使是客观把握型语言也有可能具有主观把握型的语言的特征.Arka就是一个例子.
那么,Arka具体在多大程度上混合了客观把握和主观把握呢?为此,首先要把客观把握型语言所具有的诸特征和主观把握型语言的诸特征,都列出来.
以中村(2004p41)为基础,分析了Arka的各种特征,如下表所示.
与中村(2004)、河原(2009)相比条目数有所增减,
第一行也不是I - mode、D - mode,而是采用了统一的主观把握、客观把握等,适当地进行了编辑.

如果Arka那一列是``客观'',就意味着对这一行的项目要有客观的把握.
反之,如果是``主观'',就是主观把握.
在客观>主观的情况下,可以主观把握,但表示客观更自然,主观>客观则相反.
根据列的不同,也有特殊表示,在验证栏特别记录.
\begin{table}[H]
	\centering		 
\resizebox{\textwidth}{!}{
\begin{tabular}{|c|c|c|c|} % {l|c|r}<-- Alignments: 1st column left, 2nd middle and 3rd right, with vertical lines in between
	\hline
	相同的项目&			主观把握&		客观把握&	Arka\\\hline 
	动词的理解方法&		「なる」型语言&	「する」型语言&	客观>主观\\\hline
	认知主体存在的方法&	感受者(有情者)(sentient)&	动作主(agent)&	客观>主观\\\hline 
	状况的理解方法&		「事件」「场所」型语言&	「物体」型语言&	客观\\\hline
	存在和所有&			BE语言&	HAVE语言&	客观\\\hline 
	动词的焦点&			行为中心&	结果中心&	客观>主观\\\hline 
	终结指向&			无&	有&	客观>主观\\\hline
	名词的理解方法&		无界性(unboundedness)&	有界性(boundedness)&	客观>主观\\\hline 
	名词图式&			连续体图式&	个体图式&	客观>主观\\\hline 
	第一人称代词&		多样&	一定&	主观>客观\\\hline
	敬语&				不发达的文法范畴&	敬意表现&	主观>客观\\\hline
	省略代词&			多&	少&	客观>主观\\\hline 
	非人称主语&			无&	有&	客观\\\hline 
	话题或主语&			话题优先&	主语优先&	客观\\\hline 
	连体修饰构造&		语用论的&	文法的&	客观\\\hline 
	「这里」的理解方式&	场所中心&	人中心&	特殊\\\hline  
	主客合体性&			有&	无&	客观\\\hline 
	情态表现&			认识论(epistic)&	道义论(deontic)&	客观\\\hline 
	与格还是间接宾语&	与格(利害相关)&	间接宾语(汇点)&	客观\\\hline 
	间接宾语&	有&	无&	客观\\\hline 
	(英语的)中间句法&	表现直接经验&	表现特性记述&	客观\\\hline 
	动词vs.卫星框架&	动词框架&	卫星框架&	主观\\\hline
	主观述语&			有&	无&	客观>主观\\\hline 
	拟声词和拟态词&		多&	少&	客观\\\hline 
	过去时叙述中的现在时&	多(e.g.「る」形)&	少&	特殊\\\hline 
	直接・间接叙述&		大都是直接叙述&	间接叙述也发达&	客观\\\hline 
	\end{tabular}}
\end{table}

\section{Arka的认知样式}

分析上表结果:

总项目数:12

客观:12

客观>主观:8

主观>客观:2

特殊:2

主观:1

主观胜于客观的有三项,出现率为12.0\%.由此可见,Arka具有强烈的客观把握性质.也就是说,在分类上属于英语等的同类.

但是,再深入地思考一下.像客观>主观那样不太典型的地方也算进主观的话总数就是11/25,出现率也上升到44.0\%.
除去特殊的2,计算结果是11/23,约有47.8\%的主观性被嵌入其中.
确实,客观把握和主观把握只是一种倾向.但是客观把握的语言夹杂主观把握的比例一般很低,以各占一半的比例进行把握的语言通常很难强行分类.
上面所看到的特征群作为倾向的话,就应当比较强烈才算数,例如被分类为客观把握的英语和汉语基本上都具有强客观把握的特征.

与此相比,Arka的把握方法则是客观把握和主观把握相结合.如果算上或多或少夹杂着主观性的项目,大约有一半的比例是主客交错的.
这意味着Arka有不同于日语、英语和汉语的独特的把握方法.换句话说,这意味着Arka有自己独特的先验认知方式.
本文的主题是揭示Arka的认知方式.通过对Arka独特的认知方式的理解,我们可以从上面的表格中发现阿卡掌握事物的规律.
与此同时,还能明确何为``Arka特色''.这对幻文阅读自不必说,对幻文写作也非常有用.


\section{双重把握}
\subsection{第三把握}
池上所说的客观把握和主观把握再加上第三个把握,Arka的本质就显现出来了.

客观把握是从事态之外把握事态的方式.

主观把握是从事态中把握事态的方式.

Arka将认知主体分离为主观认知主体和客观认知主体,从事态内外把握事态.这种模式被称为双重把握.

继续观念性的话题,举个例子.在足球比赛中,现在正好是进球的瞬间.如果你在观众席上看到进门的样子,那就是客观的把握.
另一方面,如果你是守门员,那就是主观把握.所谓``双重把握'',就是从两个角度来观察目标.
在电视节目中,终点的场景会以各种角度反复播放.既有观众席的影像,也有守门员视角的影像.看双方的影像就等于双重把握.

a (Arka)的前身是ar (Arvalen).ar的前身又是ls (lestir语).ls的前身又是ly (lyudia语).
\footnote{没找到英文名,日语假名分别是アルバレン:Arvalen,レスティル:lestir,リュディア:lyudia,下面的有ルティア:lutia}
Arbazad\footnote{指Arka等语言所在的世界}除了a,还有lt (lutia)和n(凪雾).凪雾与alt (Artia词)同义.
其中ly, ls, lt进行客观的把握.每一个都是以l开始的,所以也可以称为l型.
另外,n是主观把握.也可以称为N型.
a, ar具有双重把握.两者都是从a开始的,所以也可以称为a型.

sm(sermer\footnote{假名为セルメル}时代)时n进入,Arbazad的南部被征服,出现了幻凪国(karensia\footnote{カレンシア,意为金盏花}).
以此为契机,ar融入了主观把握.过去的ar继承了ly,是客观的把握,但是由于和n的融合,变成了主观的把握.其结果就是前面的双重把握.

\subsection{区分逻辑与感情,客观与主观}

Arka新手教程里Aria说过这些:

{\kaishu 对于我们阿尔巴德人来说,语言是用来---

\quad 1:组织逻辑

\quad 2:分析情绪

---的工具.特别适用于内观和主张.}

她说,Arka能巧妙地表现逻辑和感情这两个看似相反的要素.实际上,这句话体现了Arka双重把握的特征.
阿尔巴德人根据客观的把握来组织逻辑,同时根据主观的把握来分析感情.
在上表的诸多特征中,对于他们认为逻辑性比较好的东西,就用客观的把握,
对于感觉性比较好的东西,就用主观的把握.

例如,假设这里有一位女性.假设她的名字是Lydia.她在女儿面前是母亲,在丈夫面前是妻子,在上司面前是部下.
她在女儿面前可能会叫自己noel.另一方面,在丈夫面前说non,在上司面前说meid.
客观地说,她只是Lydia这个人,但在她看来,根据对方的不同,自己有时会变成non,有时会变成meid.
这时,她将自己分为主观认知主体和客观认知主体两种,从事态内外观察事态.
决定自己的称呼时,让主观认知主体发挥作用,又根据在自己眼中与对方的关系决定第一人称代词.
\footnote{译注:想想你和你的上司,一般情况下你会对你的上司出于礼貌,多少会表示尊敬,这是根据你和他的上下关系而决定的.
但是如果他真的激怒了你,你在硬怼他的时候就不会像刚才那么礼貌,这是你的主观情绪所决定的.}
Arbazard人原本只有缺乏的第一人称代词,得到了双重把握之后,就获得了通过主观地看待他人和自己的关系来选择自己称呼的新视点.其结果是a的丰富位相
\footnote{见\hyperlink{appendix-pronouns}{附录}}.

另外,给句末纯词赋予人际情态等也是主观认知主体的作用.Arka的人际情态表现在句末,从外面包住了整个句子.
与日语的终助词``ね''``さ''``よ''等相似,但不像``\~{}てしまう''那样与谓语连在一起,从这一点上看,它有将事件从外侧用人际情态包裹的特征.
也就是说,Arka对人的情态是与事件分离的(但敬语和称呼等除外).另外,关于日语的情态,庵(2001)等讲得比较明白.

\subsection{对事件进行客观的把握}

在Arka,不会用``下次我要结婚了''这样的语言表达.为了客观地看待男人和女人结婚这件事.
事件与第一人称代词不同,它只是单纯发生的事情,是客观的事情.因此要使用客观的把握.
因此ans sil mals\footnote{mals:结婚,与yul格搭配,表示对象,可省略.} im tuo(我们下次结婚)是不文的,ans sil xok
\footnote{xok:[n,adj,adv]相互.sil是"将来,将来会是",是这句的动词} im tuo(我们下次结婚)是对的.像这样,从原则上客观地看待事件.
\subsection{是「今行くよ」还是"I'm coming"}
晚饭时间,在自己房间里的自己被叫到客厅.一般来说,在客观把握的语言中,这个时候就像英语的``I'm coming''一样使用``来''.
不过需要注意的是,即使是客观把握的语言,``来''的用法也有可能与日语相同.
例如,被西班牙语称呼时,回答``我现在去''的时候,就说``Voy''.这是``Yo voy''的意思,用英语来说就是``I go''.不使用``来''.
Arka也是一样,虽然有客观把握的基础,但是关于往来的说法和西班牙语、日语一样.\footnote{总结:\quad 来:相对于客厅,去:相对于自己(现在的位置)}

但是为什么英语中要像``I'm coming''一样使用``来''呢?
在进行客观把握的语言中,认知主体处于事态的外围,因此,在自己房间里的自己已经不是认知主体了.
就像灵魂出壳一样,看着在自己房间里的自己``来到''起居室这个目的地.正因为如此,才不是going而是coming.
相反,像日语这种主观把握的语言,因为认知主体是自己房间里的自己,所以不会说``现在就来''.

Arka的情况是,在路上多使用主观把握.因为对``来''这个词更有实感的是主观把握.
例如,在足球比赛中,进球的瞬间,从观众的角度看和从守门员的角度看,哪个更能感受到球``来了''呢?当然是后者,具有动力机制.
Arbazard人对于道路,为了获得这种动力机制而使用主观把握.因此,被叫到起居室的时候不是an luna\footnote{luna:来,van:去,走} van而是an ke van.

那么,在不需要动力机制的场景中又会如何呢?也就是像新闻那样客观地捕捉往来的情况.
有趣的是,在这种情况下要使用客观把握.因此an lunat lestez成为自然.
直译的话是``我来到了起居室'',和日语的``我去了起居室''相反.如上所述,在Arka的往来在日常生活中使用主观的把握,
但在不同情况下也会使用客观的把握.因为可以根据场合分别使用,所以表现细腻.

\subsection{镜中反映的是"我"还是"你"}

对着镜子里的自己呼唤或自言自语的时候,有两种说法,一种叫自己``我'',另一种叫自己``你''.在进行客观把握的语言中,有用第二人称称呼自己的倾向.
例如说英语的时候,对自己说话的时候通常用you.当然,``What am I doing here?''虽然也有使用I的说法,
但一般客观把握的语言都倾向于客观地看待自己,用第二人称来称呼自己.
另一方面,在日语中,对自己说话的时候通常使用``我''``我''等第一人称.这是主观把握的表现.

那么Arka的情况又如何呢?在《梦织》中,少女Alis做了这样的自问自答.
``hai, non rens ax to a nain eyo?" ter, nainan !non inat adel yunen haadis vandor xe mana"?wein alis, ti lo ne xar tuube??''
``可是,我该怎么跟警察说才好呢?‘听我说,警察先生,我看到像骷髅一样的怪物袭击了一个女孩!’——对了,Alis,你觉得这种话谁会相信?''
她在自问自答的时候,把自己客观地看作``Alis''和``你''.由此可见,Arka是一种客观看待自己的语言.


\subsection{主观把握和客观把握的区别使用}
从上述例子,可以看出Arbazard人的感觉.他们认为最好结合感情的东西就会使用主观把握,否则就会继承原本的L型,使用客观把握.
这样看来,乍一看不协调、不连贯的上面的桌子上出现了一条线.原则上,上面的``Arka''一列都可以写入双重把握.归根结底,客观>主观是其中的明细.

另外,如果问到底是主观占优势还是客观占优势,Arka是客观占优势.之所以这么说,是因为原本是L型的时候受到了N型的强烈影响.
因为原本是L型,所以不管受到多大的影响,原则上L型占优势的状况是不会改变的.


\section{人类对自然的双重把握}
实际上,双重把握是人类对自然事态的认知方式.这个世界上到底有谁是靠主观判断而活的呢?
人们经常会做出这样的判断:``虽然自己不喜欢,但客观上不得不承认.''

人们在把握事态的时候常常会有双重把握,主客同时判断.然而,只在语言上偏向主观或客观的把握,本来就是奇怪的.
如果说双重把握是人类本来的认知方式,那么将其直接反映在语言结构上的,Arka的认知方式,对人类来说就可能是生理上自然的机制.

在地球发达国家几百年的历史中翻找,也难以找到一个国家或者时代,能把典型的客观把握型语言和典型的主观把握型语言进行混合,看看真的会有什么变化.
双重把握不论看起来多么自然,实际的证据仍然不充分.

在1066年的诺曼底征服中也只有英语和法语混杂在一起.在太平洋战争中日本战败,大量的美国人涌入日本,日本人也没有变成双语者.
几百年来,美国人和澳大利亚人都没有双语地使用原住民的语言.韩国虽说有过把中国作为事实上的宗主国的时代,但也不是Arbazad和karensia那样.

既然在地球上的发达国家没有像Arbazad和karensia那样大规模、长时间的语言主客混杂现象,从现实角度考虑,要获得详细的语言数据是很困难的.
因此,关于Arka的双重把握,我们只能做出``是否符合人类的认知模式'',以及``目前可以毫无问题地用这种语言进行沟通,至少不会有不自然的地方''等较弱的推测.

\section{Arka的认知语言学先验性}
Arka是先验人工语言.但也有人对人类能否从零开始创造语言提出质疑.
有人认为,即使能做出来,也只是模仿作者的母语.但是这次的分析表明,Arka拥有日语和英语都没有的独特的认知方式.
通过双重把握这一Arka独特的认知方式,证明了Arka在认知语言学上的先验性.

需要注意,``双重把握''是指可以通过主客两种方式来把握``来'',所以不是单纯的主观把握和客观把握的混合,而是像人的认知方式一样,同时把握主客这一点.

那么,在这种双重把握的认知模式的基础上,试着验证上述表格的诸特征.

{\noindent} \rule[-10pt]{17.5cm}{0.05em}
\section{验证}

">"表示左边的文字比右边的文字自然.

{\noindent} \rule[-10pt]{17.5cm}{0.05em}

\section{Arka是现代的认知样式吗?}
\section{迟到到认知语言学考察}
\section{语言相对论的再评价,以及杂感}
\subsection{认知主义和生成生成主义}
\subsection{这种``不明白''是``无法理解''还是``无法产生共鸣''}
\subsection{Arbazard人说:``理解但不共鸣.''}
\subsection{语言的影响仅限于思考的习惯}
\subsection{角色语与语言相对论}
另外,语言相对论与金水(2003,2007)所倡导的角色语也有关联。

我们来比较一下拥有丰富的角色句尾的日韩和没有的英法吧。
后者的说话者能同样感受到前者所感受到的性格吗?不论怎么说,至少关于角色句尾传达的角色性确实遗漏了。
从具有丰富第一人称代词的语言翻译成不具有丰富第一人称代词的语言时,也会出现这个问题。
举个例子吧。“わたくし、磯鷲早矢と申しますの”和“拙者、緋村剣心でござる”如果翻译成英语,
都只能是“I am Haya”或“I am Kenshin”之类的说法。

日语中有“わたくし\~{}ですわ”这样让人想起大小姐角色的词语。
日本人有根据第一人称和终助词的使用方法来观察说话人个性的思维习惯。
当然,即使是没有终助词或只有一种第一人称的语言,也可以通过其他词类和措辞来判断对方的个性。
但是,仍然没有产生用第一人称或终助词来观察对方个性的思考习惯。
从这个意义上来说,语言确实会对思考产生影响,由此可以感觉到语言相对论和作用语之间存在着关联性。

Arka和日语一样,人称代词等丰富,擅长表现人物性格。
因此,可以说Arbazard人培养了用那样的要素推测对方的人性的思考的习惯。
\section{参考文献}
Langacker, R. W.(1985) "Observations and Speculations on Subjectivity" 
Haiman (ed) "Iconicity in Syntax" pp109-150

John Benjamins Publishing Comapany――(2008) "The relevance of Cognitive Grammar for language pedagogy" 
Sabine De Knop (ed) "Cognitive Approaches to Pedagogical Grammar: A Volume in Honour of Rene Dirven 
(Applications of Cognitive Linguistics)" p18 

Mouton De Gruyter Benjamin Lee Whorf(1964)"Language, Thought, and Reality: Selected Writings of Benjamin Lee Whorf" 
John B. Carroll (ed) The MIT Press

アンナ ヴィエルジュビツカ(2009)『キーワードによる異文化理解』 p22 而立書房

スティーブン ピンカー(1995)『言語を生みだす本能〈上〉』日本放送出版協会

庵功雄(2001)『新しい日本語学入門―ことばのしくみを考える』スリーエーネットワーク

池上嘉彦(2004)「言語における〈主観性〉と〈主観性〉の言語的指標 (1)」『認知言語学論考』No.3 pp1-49 ひつじ書房――(2005)「言語における〈主観性〉と〈主観性〉の言語的指標 (2)」『認知言語学論考』No.4 pp1-60 ひつじ書房

金水敏(2003)『ヴァーチャル日本語 役割語の謎』岩波書店――(2007)『役割語研究の地平』くろしお出版

河原清志(2009)「英日語双方向の訳出行為におけるシフトの分析―認知言語類型論からの試論―」日本通訳翻訳学会・翻訳研究分科会(編)『翻訳研究への招待』第3号 pp29-49

中村芳久(2004) 「主観性の言語学:主観性と文法構造・構文」 中村芳久(編)『認知文法論Ⅱ』 pp3-51 大修館書店

森山新(2007)「認知言語学的観点による日本語の連体修飾研究-連体修飾節・ノを用いた連体修飾を中心に-」日本学報 72. pp41-58――(2009)「日本語の言語類型論的特徴がモダリティに及ぼす影響 :グローバル時代に求められる総合的日本語教育のために」比較日本学教育研究センター研究年報, 5, pp147-153
%``([a-z]+)''
%``$1''

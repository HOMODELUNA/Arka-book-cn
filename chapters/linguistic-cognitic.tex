\chapter[认知语言学层面的Arka]{认知语言学层面的Arka}
%\chapter[短标题显示在页面]{长标题显示在目录}
%Arka的\hypertarget{appendix-pronouns}{位相}

\section{客观把握与主观把握}
在认知语言学中,认知主体可以定义为对事态的把握.认知的模式可以分为两大类.

一种是从事态的外部把握事态的样式,相当于池上(2002,2005)
\footnote{池上嘉彦(2004)「言語における〈主观性〉と〈主观性〉の言語的指標 (1)」『认知言語学論考』No.3 pp1-49 ひつじ書房
――(2005)「言語における〈主观性〉と〈主观性〉の言語的指標 (2)」『认知言語学論考』No.4 pp1-60 ひつじ書房}所说的客观把握.

一种是从事态内部把握事态的方式,同样相当于主观把握.

事实上同一概念,语言学家用各自的术语来表示.
例如,Langacker(1985)将客观把握称为最佳视点阵列(optimal viewing arrangement),中村(2004)设为D模式.
同样是主观把握,Langacker用自我中心的视点排列(egocentric viewing arrangement),中村用I模式.
本文以最容易从字面理解意思为由,模仿池上的术语.

在语言学上并不矛盾的人工语言的制作方法中,``做"型语言(する语言)相当于客观的把握,"成为''型语言(なる语言)相当于主观的把握.
前文中所涉及的特征数量很少,如做·成为,物·事、have·be等,主要是用``做"和"成为''的对比来说明的.
但是,由于本文所涉及的特征数量太多,所以用する语言、なる语言反而会变得晦涩难懂.因此,从这里开始,要把术语统一为客观把握和主观把握.

当然,各个术语的意思各不相同.
例如中村(2004p40)将I - mode称为``西田的认识论'',将D - mode称为``笛卡尔的认识论'',从哲学角度进行了分析.
在这个意义上,术语的语感各不相同,本文为了方便,统一使用池上的术语.

\section{Arka掌握事态的方法}

在前面的文章中,我们已经明确Arka具有客观把握和主观把握两方面的特征.
森山(2009)提出:一般来说,语言的认知主体进行客观把握,很少夹杂主观把握.

将外界事态语言化有``客观把握''和``主观把握''两种把握方法,根据语言的不同,选择哪一种优先,左右着每种语言所具有的整体类型论特征.
但客观把握和主观把握不是规律,而是倾向,所以即使是客观把握型语言也有可能具有主观把握型的语言的特征.Arka就是一个例子.
那么,Arka具体在多大程度上混合了客观把握和主观把握呢?为此,首先要把客观把握型语言所具有的诸特征和主观把握型语言的诸特征,都列出来.
以中村(2004p41)为基础,分析了Arka的各种特征,如下表所示.
与中村(2004)、河原(2009)相比条目数有所增减,
第一行也不是I - mode、D - mode,而是采用了统一的主观把握、客观把握等,适当地进行了编辑.

如果Arka那一列是``客观'',就意味着对这一行的项目要有客观的把握.
反之,如果是``主观'',就是主观把握.
在客观>主观的情况下,可以主观把握,但表示客观更自然,主观>客观则相反.
根据列的不同,也有特殊表示,在验证栏特别记录.
\begin{table}[H]
	\centering		 
\resizebox{\textwidth}{!}{
\begin{tabular}{|c|c|c|c|} % {l|c|r}<-- Alignments: 1st column left, 2nd middle and 3rd right, with vertical lines in between
	\hline
	相同的项目&			主观把握&		客观把握&	Arka\\\hline 
	动词的理解方法&		「なる」型语言&	「する」型语言&	客观>主观\\\hline
	认知主体存在的方法&	感受者(有情者)(sentient)&	动作主(agent)&	客观>主观\\\hline 
	状况的理解方法&		「事件」「场所」型语言&	「物体」型语言&	客观\\\hline
	存在和所有&			BE语言&	HAVE语言&	客观\\\hline 
	动词的焦点&			行为中心&	结果中心&	客观>主观\\\hline 
	终结指向&			无&	有&	客观>主观\\\hline
	名词的理解方法&		无界性(unboundedness)&	有界性(boundedness)&	客观>主观\\\hline 
	名词图式&			连续体图式&	个体图式&	客观>主观\\\hline 
	第第一人称代词&		多样&	一定&	主观>客观\\\hline
	敬语&				不发达的文法范畴&	敬意表现&	主观>客观\\\hline
	省略代词&			多&	少&	客观>主观\\\hline 
	非人称主语&			无&	有&	客观\\\hline 
	话题或主语&			话题优先&	主语优先&	客观\\\hline 
	连体修饰构造&		语用论的&	文法的&	客观\\\hline 
	「这里」的理解方式&	场所中心&	人中心&	特殊\\\hline  
	主客合体性&			有&	无&	客观\\\hline 
	情态表现&			认识论(epistic)&	道义论(deontic)&	客观\\\hline 
	与格还是间接宾语&	与格(利害相关)&	间接宾语(汇点)&	客观\\\hline 
	间接宾语&	有&	无&	客观\\\hline 
	(英语的)中间句法&	表现直接经验&	表现特性记述&	客观\\\hline 
	动词vs.卫星框架&	动词框架&	卫星框架&	主观\\\hline
	主观述语&			有&	无&	客观>主观\\\hline 
	拟声词和拟态词&		多&	少&	客观\\\hline 
	过去时叙述中的现在时&	多(e.g.「る」形)&	少&	特殊\\\hline 
	直接/间接叙述&		大都是直接叙述&	间接叙述也发达&	客观\\\hline 
	\end{tabular}}
\end{table}

\section{Arka的认知样式}

分析上表结果:

总项目数:12

客观:12

客观>主观:8

主观>客观:2

特殊:2

主观:1

主观胜于客观的有三项,出现率为12.0\%.由此可见,Arka具有强烈的客观把握性质.也就是说,在分类上属于英语等的同类.

但是,再深入地思考一下.像客观>主观那样不太典型的地方也算进主观的话总数就是11/25,出现率也上升到44.0\%.
除去特殊的2,计算结果是11/23,约有47.8\%的主观性被嵌入其中.
确实,客观把握和主观把握只是一种倾向.但是客观把握的语言夹杂主观把握的比例一般很低,以各占一半的比例进行把握的语言通常很难强行分类.
上面所看到的特征群作为倾向的话,就应当比较强烈才算数,例如被分类为客观把握的英语和汉语基本上都具有强客观把握的特征.

与此相比,Arka的把握方法则是客观把握和主观把握相结合.如果算上或多或少夹杂着主观性的项目,大约有一半的比例是主客交错的.
这意味着Arka有不同于日语、英语和汉语的独特的把握方法.换句话说,这意味着Arka有自己独特的先验认知方式.
本文的主题是揭示Arka的认知方式.通过对Arka独特的认知方式的理解,我们可以从上面的表格中发现Arka掌握事物的规律.
与此同时,还能明确何为``Arka特色''.这对幻文阅读自不必说,对幻文写作也非常有用.


\section{双重把握}
\subsection{第三把握}
池上所说的客观把握和主观把握再加上第三个把握,Arka的本质就显现出来了.

客观把握是从事态之外把握事态的方式.

主观把握是从事态中把握事态的方式.

Arka将认知主体分离为主观认知主体和客观认知主体,从事态内外把握事态.这种模式被称为双重把握.

继续观念性的话题,举个例子.在足球比赛中,现在正好是进球的瞬间.如果你在观众席上看到进门的样子,那就是客观的把握.
另一方面,如果你是守门员,那就是主观把握.所谓``双重把握'',就是从两个角度来观察目标.
在电视节目中,终点的场景会以各种角度反复播放.既有观众席的影像,也有守门员视角的影像.看双方的影像就等于双重把握.

a (Arka)的前身是ar (Arvalen).ar的前身又是ls (lestir语).ls的前身又是ly (lyudia语).
\footnote{没找到英文名,日语假名分别是アルバレン:Arvalen,レスティル:lestir,リュディア:lyudia,下面的有ルティア:lutia}
Arbazad\footnote{アルバザード,指Arka等语言所在的世界}除了a,还有lt (lutia)和n(凪雾).凪雾与alt (Artia词)同义.
其中ly, ls, lt进行客观的把握.每一个都是以l开始的,所以也可以称为l型.
另外,n是主观把握.也可以称为N型.
a, ar具有双重把握.两者都是从a开始的,所以也可以称为a型.

sm(sermer\footnote{假名为セルメル}时代)时n进入,Arbazad的南部被征服,出现了幻凪国(karensia\footnote{カレンシア,意为金盏花}).
以此为契机,ar融入了主观把握.过去的ar继承了ly,是客观的把握,但是由于和n的融合,变成了主观的把握.其结果就是前面的双重把握.

\subsection{区分逻辑与感情,客观与主观}

Arka新手教程里Aria说过这些:

{\kaishu 对于我们Arbazard人来说,语言是用来---

\quad 1:组织逻辑

\quad 2:分析情绪

---的工具.特别适用于内观和主张.}

她说,Arka能巧妙地表现逻辑和感情这两个看似相反的要素.实际上,这句话体现了Arka双重把握的特征.
Arbazard人根据客观的把握来组织逻辑,同时根据主观的把握来分析感情.
在上表的诸多特征中,对于他们认为逻辑性比较好的东西,就用客观的把握,
对于感觉性比较好的东西,就用主观的把握.

例如,假设这里有一位女性.假设她的名字是Lydia.她在女儿面前是母亲,在丈夫面前是妻子,在上司面前是部下.
她在女儿面前可能会叫自己noel.另一方面,在丈夫面前说non,在上司面前说meid.
客观地说,她只是Lydia这个人,但在她看来,根据对方的不同,自己有时会变成non,有时会变成meid.
这时,她将自己分为主观认知主体和客观认知主体两种,从事态内外观察事态.
决定自己的称呼时,让主观认知主体发挥作用,又根据在自己眼中与对方的关系决定第第一人称代词.
\footnote{译注:想想你和你的上司,一般情况下你会对你的上司出于礼貌,多少会表示尊敬,这是根据你和他的上下关系而决定的.
但是如果他真的激怒了你,你在硬怼他的时候就不会像刚才那么礼貌,这是你的主观情绪所决定的.}
Arbazard人原本只有缺乏的第第一人称代词,得到了双重把握之后,就获得了通过主观地看待他人和自己的关系来选择自己称呼的新视点.其结果是a的丰富位相
\footnote{见\hyperlink{appendix-pronouns}{附录}}.

另外,给句末纯词赋予人际情态等也是主观认知主体的作用.Arka的人际情态表现在句末,从外面包住了整个句子.
与日语的终助词``ね''``さ''``よ''等相似,但不像``\~{}てしまう''那样与谓语连在一起,从这一点上看,它有将事件从外侧用人际情态包裹的特征.
也就是说,Arka对人的情态是与事件分离的(但敬语和称呼等除外).另外,关于日语的情态,庵(2001)等讲得比较明白.

\subsection{对事件进行客观的把握}

在Arka,不会用``下次我要结婚了''这样的语言表达.为了客观地看待男人和女人结婚这件事.
事件与第第一人称代词不同,它只是单纯发生的事情,是客观的事情.因此要使用客观的把握.
因此ans sil mals\footnote{mals:结婚,与yul格搭配,表示对象,可省略.} im tuo(我们下次结婚)是不文的,ans sil xok
\footnote{xok:[n,adj,adv]相互.sil是``将来,将来会是'',是这句的动词} im tuo(我们下次结婚)是对的.像这样,从原则上客观地看待事件.
\subsection{是「今行くよ」还是``I'm coming''}
晚饭时间,在自己房间里的自己被叫到客厅.一般来说,在客观把握的语言中,这个时候就像英语的``I'm coming''一样使用``来''.
不过需要注意的是,即使是客观把握的语言,``来''的用法也有可能与日语相同.
例如,被西班牙语称呼时,回答``我现在去''的时候,就说``Voy''.这是``Yo voy''的意思,用英语来说就是``I go''.不使用``来''.
Arka也是一样,虽然有客观把握的基础,但是关于往来的说法和西班牙语、日语一样.\footnote{总结:\quad 来:相对于客厅,去:相对于自己(现在的位置)}

但是为什么英语中要像``I'm coming''一样使用``来''呢?
在进行客观把握的语言中,认知主体处于事态的外围,因此,在自己房间里的自己已经不是认知主体了.
就像灵魂出壳一样,看着在自己房间里的自己``来到''起居室这个目的地.正因为如此,才不是going而是coming.
相反,像日语这种主观把握的语言,因为认知主体是自己房间里的自己,所以不会说``现在就来''.

Arka的情况是,在路上多使用主观把握.因为对``来''这个词更有实感的是主观把握.
例如,在足球比赛中,进球的瞬间,从观众的角度看和从守门员的角度看,哪个更能感受到球``来了''呢?当然是后者,具有动力机制.
Arbazard人对于道路,为了获得这种动力机制而使用主观把握.因此,被叫到起居室的时候不是an luna\footnote{luna:来,van:去,走} van而是an ke van.

那么,在不需要动力机制的场景中又会如何呢?也就是像新闻那样客观地捕捉往来的情况.
有趣的是,在这种情况下要使用客观把握.因此an lunat lestez成为自然.
直译的话是``我来到了起居室'',和日语的``我去了起居室''相反.如上所述,在Arka的往来在日常生活中使用主观的把握,
但在不同情况下也会使用客观的把握.因为可以根据场合分别使用,所以表现细腻.

\subsection{镜中反映的是``我"还是"你''}

对着镜子里的自己呼唤或自言自语的时候,有两种说法,一种叫自己``我'',另一种叫自己``你''.在进行客观把握的语言中,有用第二人称称呼自己的倾向.
例如说英语的时候,对自己说话的时候通常用you.当然,``What am I doing here?''虽然也有使用I的说法,
但一般客观把握的语言都倾向于客观地看待自己,用第二人称来称呼自己.
另一方面,在日语中,对自己说话的时候通常使用``我''``我''等第一人称.这是主观把握的表现.

那么Arka的情况又如何呢?在《梦织》中,少女Alis做了这样的自问自答.
{\kaishu 
``hai, non rens ax to a nain eyo?`` ter, nainan !non inat adel yunen haadis vandor xe mana''?wein alis, ti lo ne xar tuube??''

``可是,我该怎么跟警察说才好呢?‘听我说,警察先生,我看到像骷髅一样的怪物袭击了一个女孩!’——对了,Alis,你觉得这种话谁会相信?''
}

她在自问自答的时候,把自己客观地看作``Alis''和``你''.由此可见,Arka是一种客观看待自己的语言.


\subsection{主观把握和客观把握的区别使用}
从上述例子,可以看出Arbazard人的感觉.他们认为最好结合感情的东西就会使用主观把握,否则就会继承原本的L型,使用客观把握.
这样看来,乍一看不协调、不连贯的上面的桌子上出现了一条线.原则上,上面的``Arka''一列都可以写入双重把握.归根结底,客观>主观是其中的明细.

另外,如果问到底是主观占优势还是客观占优势,Arka是客观占优势.之所以这么说,是因为原本是L型的时候受到了N型的强烈影响.
因为原本是L型,所以不管受到多大的影响,原则上L型占优势的状况是不会改变的.


\section{人类对自然的双重把握}
实际上,双重把握是人类对自然事态的认知方式.这个世界上到底有谁是靠主观判断而活的呢?
人们经常会做出这样的判断:``虽然自己不喜欢,但客观上不得不承认.''

人们在把握事态的时候常常会有双重把握,主客同时判断.然而,只在语言上偏向主观或客观的把握,本来就是奇怪的.
如果说双重把握是人类本来的认知方式,那么将其直接反映在语言结构上的,Arka的认知方式,对人类来说就可能是生理上自然的机制.

在地球发达国家几百年的历史中翻找,也难以找到一个国家或者时代,能把典型的客观把握型语言和典型的主观把握型语言进行混合,看看真的会有什么变化.
双重把握不论看起来多么自然,实际的证据仍然不充分.

在1066年的诺曼底征服中也只有英语和法语混杂在一起.在太平洋战争中日本战败,大量的美国人涌入日本,日本人也没有变成双语者.
几百年来,美国人和澳大利亚人都没有双语地使用原住民的语言.韩国虽说有过把中国作为事实上的宗主国的时代,但也不是Arbazad和karensia那样.

既然在地球上的发达国家没有像Arbazad和karensia那样大规模、长时间的语言主客混杂现象,从现实角度考虑,要获得详细的语言数据是很困难的.
因此,关于Arka的双重把握,我们只能做出``是否符合人类的认知模式'',以及``目前可以毫无问题地用这种语言进行沟通,至少不会有不自然的地方''等较弱的推测.

\section{Arka的认知语言学先验性}
Arka是先验人工语言.但也有人对人类能否从零开始创造语言提出质疑.
有人认为,即使能做出来,也只是模仿作者的母语.但是这次的分析表明,Arka拥有日语和英语都没有的独特的认知方式.
通过双重把握这一Arka独特的认知方式,证明了Arka在认知语言学上的先验性.

需要注意,``双重把握''是指可以通过主客两种方式来把握``来'',所以不是单纯的主观把握和客观把握的混合,而是像人的认知方式一样,同时把握主客这一点.

那么,在这种双重把握的认知模式的基础上,试着验证上述表格的诸特征.

{\noindent} \rule[-10pt]{17.5cm}{0.05em}
\section{验证}

``>''表示左边的文字比右边的文字自然.

\subsection{动词理解方法:kegr>主观}

\subsubsection{スル还是ナル}

la fian pixat gek > gek at pix:动作主明确的情况下自然使用他动词.
la vals kea sil ti > ti sil kea mil la vals
但是主语是le pita这样的无生物的情况下,ti sil kea mil le pita更占优势.因此可以判定为客观>主观.

\subsubsection{行为连锁(action chain)}

英:原因→动作主→手段→对象

This medicine(原因) will make you(对象) better.
Tom(动作主) broke the wall.(对象)
The hammer(手段) broke the wall(对象).

幻:动作主→对象

?? tu pita(原因) kea sil ti(对象). → ti sil kea mil tu pita.
arxe(动作主) rigat bal(对象).
?? bolt(手段) rigat bal(对象). → bal at rig mil(kon) bolt. = xe rigat bal kon bolt.

Arka的主语sol原本是“做~者”的意思,所以原则上主语是agent.
但是被动句的时候sol和宾语yul是互换的,所以不在此限.另外,动词为et、em、sil、ses、at、or等定义动词时,经验者作为主语.
除此之外的原因和手段等用斜格表示.

\subsection{认知主体存在的方式:客观>主观}

\subsubsection{客观地把握事件}

如上所述,在拥有双重把握的arca中,对于与感情无关的单纯客观事件,会使用客观把握.
lana twal rsit dajna emil e(不能因目的而让路).直译过来就是“你的目的是为了让我的热情发挥作用”.
完全是客观的把握.如上所述,当动词各项的语感较低时,无生物主语是自然的.因此,Arka本来就具有很强的客观把握的性质.
而在日语中,与之形成鲜明对比的是,
特意将“(私は)(あなたの)目的次第では道を譲れない((我)不能因(你的)目的而让道)”的主语当作主语来主观地思考.
和日语比较的话就很清楚区别了.

\subsubsection{主语,生命度(animacy)与焦点化}


确实,事件是客观把握的,但有``主语sol就是agent''的原则.因此,正如在行为连锁中所述,主语通常不是原因或手段.
agent通常是有生的,所以生命度越高的东西越容易成为主语.我们来看下面的例子.

ak ti lunat tuube sern? > to piot ti a tuube sern?

to(什么)比ti(你)的生命度低,前者就会更自然.

日:どうしてこんな案になったの?

英:How did you get this idea? / What led you to this idea?

汉:你怎么想到这个主意的? / 什么让你想到这个主意的?

幻:ak ti lunat tuube sern? / ??to piot ti a tuube sern?

这样作为原则sol是生命度高的动作主和经验者,但是lana twal rsit dajna emil e中sol是无生物.

这是由于其他格中没有生命度高的.如果宾格或斜格中有生命度高的,它就会优先放在主语的位置上.

比如:\quad ema rest lana twal > lana twal rsit ema.

但是,当认知主体聚焦特定的无生物时,即使是无生物也可以放在主语的位置上.
聚焦于“你的目的”时,ema rest lana twal < lana twal rsit ema.

\subsection{状況的捕捉方式:客观}

如下述所示,Arka是一种``事(コト)''语言.

*ti siina vis ant?(你喜欢我吗?(私のこと好き? 指相关的事件))

ti siina an?(你喜欢我吗?(私を好き? 指人本身))

但是,Arka有英语中没有的事与物的区别,从这一点来看,带有微弱的主观把握的性质.
英语中的thing没有事与物的区别.而Arka有fam、vis、tul三个单词.

顺便说一下,能够区别这些的是arka, f\_ar之前全部用al来表示.很明显是alt的影响.

\subsection{存在还是所有:客观}

如下述所示,Arka是have型语言.

*amel xa an(妹妹在我的地方) 文意会改变,所以不可.

an til amel(我有个妹妹)

\subsection{动词的焦点/终结指向:客观>主观}

正如在语言学上并不矛盾的人工语言的制作方法中所述,Arka的动词并不意味着行为的完成.
在这一点上进行主观把握.
但是,这里面附加了一个条件,“如果没有附带条件,就视为完成”.
也就是说,只要说到an sosot la,之后不加上tal的附言,原则上就意味着行为的完成.
在这一点上,Arka还是有客观的把握的.
实际上,Arka对于动词的焦点有着非常特殊的逻辑结构.
不仅客观把握的性质,同时还比日语更具有主观把握的性质.以下举例.

在主观把握的日语中,“说服了他,但他不听”是很自然的\footnote{各位汉语着想想,是不是这个句子都非常诡异,这就是汉语作为客观把握型语言的一个表现}
,但“杀了他,但他没死”给人一种违和感.但如果是Arka,an setat la tal la en vort就很自然了.
如果以an setat la结束,没有附带说明的话,似乎是客观的把握,他在心照不宣中死去了.
但是,只要加上tal,任何内容都可以自然地推翻,
连“杀了但没死”这种主观把握的日语也感到违和感的句子变得自然.
在这一点上,Arka动词的焦点是“原则上是客观的把握,但如果加上备注的话,主观把握的性质就极其强烈”,具有特殊的性质.
因此,结论是客观>主观.因此,关于结束指向性,在无标处是“有”,如果加上备注则变为“无”.

另外,需要注意的是,过去时的无标表示完成,而现在时和将来时的无标表示不定.
在an soso la不知道他是否被说服了,原本这是近未来的表现,现在可能连说服行为都没有进行.

\subsection{名词的捕捉方式/名词的图式(scheme):客观>主观}

\subsubsection{Arka的名词在无标状态下表示“个数或是概念”}

简单来说,如果名词有复数和可数不可数,就可以看作有界性或有个体图式(scheme).为了避免烦琐,下文将术语归纳为“有界性”和“无界性”.
客观把握的情况一般有有界性,ly, ls, lt等属于有界性.英语不用说就有界性,在中文语料库的调查中,``我想要苹果"比"我想要一个苹果''更自然.
因此,是表意文字还是表音文字与有无有界性没有关系.

ar继承ls具有有界性,但在alt的影响下逐渐有条件地带有无界性.那就是“名词如果是单数的话原则上省略个数”.但这样不能只提取名词的概念.
无法区分是“苹果”的概念还是“一个具体的苹果”.因此,在ar中,即使是单数名词,加上“一个”这个词的比例还是很高的.
在a中,无界性进一步发展,“名词如果是单数,原则上省略个数”成为了普遍现象.没有概念和个数的区别,必要时用ko miik、vei miik等数字表示个数.
另外,ly, ls, ar, a在名词的形态上始终没有单复数的区别.也就是说,不存在apple和apples那样的对立.这是使用作为表意文字的幼字的f, fv的残留,与中文等相同.

综上所述,Arka名词在无标状态下表示“个数或是概念”.既不能说是有界性也不能说是无界性,但是因为有原则单数这个标准,所以可以说是倾向于有界性的.
\subsubsection{猫还是猫肉}

有界性比Arka更清晰的英语是I like cats, but I don't like cat(喜欢猫但不喜欢猫肉).与之相对,an siina ket tal an en siina ket是意思不明的句子.
必须是an siina ket tal an en siina yek e ket.在这一点上倾向于无界性.

\subsubsection{无界性的贫乏程度如何}

那么,Arka有日语那种主观把握的无界性吗?即主观>是否存在判定为客观的无界性?从结论来说没有.因此客观>可以判定为主观.
日语中说“有虫”的时候,虫不一定是一只.这里可以看到名词的无界性.另一方面,在Arka说到veliz xa atu时,只有一只虫子,这一点和日语不同.

另外,在日语中,即使买了3个苹果,也有“买了3个苹果左右”的情况.从逻辑上考虑是不自然的,但是作为日语的句子是很自然的.
Arka没有这种感觉的无界性.变成an taut vi miik.an taut vi via miik就像字面意思一样,本人不记得买了多少,只有在不确定是不是3的时候才能使用.在这一点上,Arka的原则——客观把握的性质发挥了作用.

另外,日语中经常使用的“など”、“とか”、“でも”,用Arka来表示的话,经常被省略.加上wen的场合只在想要提及字面意思中陈述的名词以外的存在时使用.这也显示了Arka名词的有界性.
另外,日语中的“など”和“とか”在很难说清楚的时候有模糊的效果.
例如香蕉很便宜,所以很容易说“バナナが食べたい(想吃香蕉)”,但是桃子很贵,所以有时会使用“できればモモとかが欲しいんだけどなぁ(如果可以的话想要桃子什么的)”这样的“とか(之类)”.
这个时候,并不是说“モモとか(桃子)”就可以用葡萄吗?说到底,本人不是桃子是不行的.
有这样深奥的表达的日语很方便,Arka也可以有同样的表达.只要给名词加上暧昧化的后缀te就可以了.如non xen lan diaikte aan.

\subsubsection{0只猫}

一般来说,越是有界性的语言,越倾向于使用nothing、no apples等零的表达方式.
因为Arka本来就有很强的有界性,所以an inat yuu ket > an en inat ket.

\subsubsection{指示代词}

我想在读Arka的小说的时候,会看到最初的lu fian渐渐变成fian.
这并不是指示代词的省略,反而意味着那个少女的固有名词化,简单来说就是“角色化”.变成fian的瞬间,那意味着“因为还不知道名字,姑且暂时把这个角色叫做‘少女’吧”.专有名词不需要指示代词,所以lu会掉下来.
这样看来,乍一看Arka的名词无标似乎必须修改为“是个体、概念还是固有名词”.但是固有名词包含在个人事物中,所以不需要规则的变更.

\subsection{第一人称代词:主观>客观}

\subsubsection{丰富的位相}

a受到alt的影响,相位发达了,其中第一人称代词也变得多样化.现在已经发展到超过alt的程度,拥有12种第一人称代词.

\subsubsection{称呼自己名字的情况和称呼长辈的情况}

一方面有典型的第一人称代词,另一方面也继续保持原本的客观把握.例如,Seren被身为长辈的莉萨叫去找她.
这时用主观把握来表现的话,就是men ket liiza xanxa(我去了老师那里).
另一方面,用客观的把握来表现的话,就是seren lunat liiza(Seren到了liza那里).
关于往来和第一人称代词等几个特征,Arka会优先主观把握,但在必须客观把握事态时,则会优先客观把握.
例如在公共场合,或者在战斗中这样紧张的场合,客观的把握是优先的.
在日语中,称呼自己的名字给人一种主观、幼稚的印象,而用Arka称呼自己的名字则是一种客观把握的情况,倒不如说是一种庄重的表现.

实际上,现实中的古Arka对名字有主观把握的倾向.因为Mer一直称自己为Mer,而不是non,这显示了他的幼稚.总觉得她是为了迎合Seren的喜好才用自己的名字称呼自己的.
与此同时,古Arka有上述“客观把握时自己也叫名字,长辈也直呼其名”的习惯,所以Seren有时也直呼liza.
也就是说,在古Arka称呼自己的时候,像Mer这样表现出幼稚的主观把握和“Seren到liza那里来了”这样的客观把握是未分化的.因此,只能根据称呼自己是幼稚还是客观来判断.

2011年的现在,虽然紫亚还很年幼,但她说自己是non或yuna,所以在称呼自己的时候就会表现出年幼的样子,这种主观的把握已经消失了.
另外,她在指示哥哥jult玩的时候``xianyan luna soa kont yuutxan…''(因为紫亚有小优,所以小优呢~)之类的话.
这在语法上是客观的把握,但在xianyan等单词层面上来说,也可以看成是主观的把握.是无法判断的地方.
顺便说一下,关于Mer,在2000年代后半期不知不觉中不再称呼自己的名字了(日语中至今仍经常称呼自己为Mer).
在成为新生之后,主观把握的用法似乎消失了.如果说Mer叫自己名字的原因是对Seren的喜好,那就是日语的影响.
但是新生的Arka自然地消除了这种影响.也就是说,为了保持先验性的自净作用在起作用.

再次总结一下,在Arka用名字称呼自己并不是幼稚的表现,而是客观把握的表现.
Arbazard人在开会等的时候会把这个搞得多种多样.例如“中午12点爱丽丝来大学,13:00点Shuba把鼓搬到活动室”,在讨论作业工序的时候避开第一人称代词,喜欢使用固有名词来进行客观把握.
像这样根据情况分别使用主观和客观,从比例上来说,主观居多.因此,可以判定主观>客观.

\subsection{敬语:主观>客观}

一般来说,主观把握的语言会用自己的视角来观察对方,与客观性语言用鸟瞰图来观察人际关系相比,更容易在意上下级关系,敬语就很容易发展.
在主观把握的语言中,有把敬语确立为语法范畴的语言.日语、韩语、爪哇语、泰语、高棉语等.
那么,客观把握事物的语言中有没有与敬语相对应的内容呢?与日语中的敬语相当的东西确实存在.
只不过,与主观把握的语言相比,它在语法上还没有体系化的东西很多,总的说来更多的是一种敬意表达.

Arka在人际关系等,在感情优先于逻辑的东西上有使用主观把握的倾向,所以和第一人称代词一样,在敬语方面主观把握占优.
礼貌语在形态论层面已经体系化,尊敬语和谦让语在总体层面已经体系化.
{\kaishu

尊敬语:kul(話す) → mist kul(お話になる)

丁宁语:tisee(~だよ) → antisee(~ですよ)

谦让语:kul(話す) → mir yuus men kul(お話しする、私から話をさせていただく)
}

也有包含尊敬和丁宁意味的单词.
{\kaishu

ku(言う) → rens(仰る)

skin(座る) → fians(お座りになる)

mok(寝る) → xidia(お休みになる)
}

但是右侧的单词本来是女性语或雅语,要想真正意义上成为尊敬语,就必须加上mist,如mist rens.

虽然敬语如此发达,但在客观把握的情况下绝对不会使用敬语.
以日语为例,如果社长是山田,那么在会议中,普通职员不可能对社长本人说“山田15:00点会来铃木工务店”.
然而,在Arka这种需要客观把握的场景中,不客观陈述反而会显得失礼.所谓失礼,就是没有认真做事,给人一种随便的感觉.
在这一点上,Arka有客观的把握.所以判断是主观>客观.

\subsection{代词の省略:客观>主观}

代词有省略.但是有条件.比英语多但比日语少.

(1)重句的主语一致:an ket felka, felat.(我去学校学习了)/ leevat felka, an kuit mar ka sea.(出了学校,在商场吃了甜甜圈)

(2)主句和从句的主语一致:an kut soa im in la.(我看到他的时候是这么说的)

(3)和前面句子的主语相同

(4)通过上下文可以明确判断

只有(4)比日语严格,不想日语一样高度依赖语境.

在故事和对话中登场人物较少的场景中会反复使用特定的代词.因为这个看起来繁琐而且不完美,所以有时会适应(3)和(4).

关于(1),在任何场合都可以使用,在英语中也经常出现,如“I went to the store and bought some brown sugar”.
关于(3)(4)相同代词的出现是在既不繁琐又不难看的时候发生的,所以虽然可以看出日语的性质,但其频度始终是有限的.

从以上判断,客观>主观.

\subsection{非人称主语:客观}

注)客观把握有非人称主语(非人称句法).但是,从动词的活用可以看出主语代词的西班牙语等的情况下,因动词的活用而成为非人称主语的代词不会出现.

\subsubsection{形式主语(虚辞)}

tu et rat xel ti ke felka(你去学校是好事)成立,因此存在非人称主语\footnote{想想英语中的``It is good that...'',it就是形式主语}.因此被判定为客观.

\subsubsection{自然现象}

It snows a lot during winter和We have a lot of snow this winter都很自然.另一方面,Arka又如何呢?

? sae ati di fol xier:因为sae看起来不像动词而带有``?".saet既然有了过去形,就去掉"?''.
sae luna ati\{du\} di fol xier.
? ans til sae di fol xier

We have a lot of snow this winter中的We不是指具体在场的我们,而是指住在那个国家或地区的模糊的人们.用法接近总称they,是形式性的主语.
Arka的情况下,总称使用el,所以el til sae di fol xier是很自然的,但ans的意图就变了.
ans有具体的“我们”的意思,所以只在夸耀祖国,与对方的地区比较时,说“我们经常下雪”.
因此,Arka对于非人称主语有客观的把握,但是需要注意总称的用法,要注意适当使用el而不是ans和laas.

\subsubsection{天气之神}

实际上,在Arka的句子中,sae、esk等自然现象句并不使用非人称主语.
因为在Kaldia有天气之神kleevel,所以动词esk的真正意思是kleevel神在宾语的地方下雨”.只是省略了kleevel,变成了eskat im fis.因此,严格来说自然现象句不是非人称主语.

\subsection{话题还是主语:客观}

关于这个特征,可以归结为“僕はウナギだ
\footnote{译注:字面理解就是``我是鳗鱼",但在日语里面"は"表示话题,那么还可能有别的含义,比如下文的"我点的是鳗鱼''.}”这句话能不能说的问题.
Arka乍一看sol是主题.实际上,在语法论中,为了方便起见,有时甚至会说明主题.
用``an et beska"或"an et har''表示“我点的是鳗鱼”或“我穿的是红色的”等意思.所以看起来是主观把握,其实不然.
因为这原本是``an et les retat beska"和"an et les sabes lein har''等的缩写.因此判定是客观的.
原本Arka中表示主语的sol是“做者”的意思,原则上主语是agent.不是题目.

\subsection{连体修饰构造:客观}

森山(2007)提``「不论是用连体修饰节,还是of或"ノ''进行修饰,英语的修饰总要包含空间,逻辑,文法之类本质的关系(intrinsic relationship).
相对地日语在这种用法以外,还会包含场景,语境,语用上的推论,依存于语用的修饰也明显发达."

另一方面,根据Langacker(2008)的说法,
“Note that we say the color of the lawn but the brown spot in (*of) my lawn, 
the difference being that the spot is not supposed to be there”.
将这个例子Arka对照如下.

nim e kist(草坪的颜色)

boppo lette kaen\{?e\} kist(草坪棕色的部分)

日语的``の"和"AのB''形结构中,A与B关系在语用上成立换言之需要依靠常识和语境去理解.

另一方面,英语的修饰至少需要空间的/文法的关系来理解.以森山所说的``本质的关系'',就不应该用of,而是用in.
在“语法的理解”这一意义上,英语的连缀修饰结构相对于语用论来说可以称为语法的.
从Arka的例子来看,从语法方面理解连体修饰更为恰当.因此判定是客观的.

\subsection{``这里''的理解方式:特殊}

日:ここはどこですか / (?)私はどこにいますか / (??)これはどこですか
汉:这里是哪里?/我在哪里?/*这个在哪?	(只能是指着地图询问的时候说得过去)
英:*Where is here? / Where am I? / *Where is this?(同汉语)
幻:(?)atu et am? / (?)an xa am? / tu et am?

比较一下前头没有``?''的自然句子,就能发现Arka不能分为日语和英语的任何一类.
既能客观地把握英语的an xa am,也能主观地把握日语的atu et am.话虽如此,但无论哪一种都不自然.所以判定为特殊.
但是,总的说来tu et am?本身是接近日语的表现吧.在这种情况下,也许可以认为是客观<主观.
\subsection{主客合一性:客观}

我们来对照一下这类话题常用的例句.
{\kaishu
日:国境の長いトンネルを抜けると雪国であった.

汉:列车穿过长长的隧道,跨过国境,到达了雪国.}
\footnote{句子出处为川端康成的《雪国》.注意此处日语和汉语的区别,日语里没有明显地出现``列车",因为说话者此时和列车是"合而为一的'',
汉语和英语都显式地提出了``列车"这一观察客体.译者尝试仿照日语句子,说"(?)穿过国境上的长长的隧道,就到达了雪国'',但总是感觉不对劲.
}
{\kaishu
英:The train came out of the long tunnel into the snow country.
幻:tu lop lukok fia e sae xi lof fil kaen kaddirei.
}
Arka和英语同属客观,没有主客合一性.

\subsection{模态表现:客观}

模态分为epistemic、deontic、evidential、dynamic等种类.

epistemic是表示推论的模态,“可能”“肯定”等都属于这种情况.在英语中may、must、will等都属于这种情况.

deontic是表示“应该”、“必须”等的情态.英语中也使用may和must.

注意到,may和must既可以用于epistemic,也可以用于deontic.词义广泛.那么,哪一个才是真正的呢,那就是deontic.
“也许吧”之类的推论终究只是自己主观的判断.另一方面,“必须做”是客观的义务.英语为了客观地把握,首先有deontic的情态must.
在此基础上派生出“一定”的epistemic语义.

那么,Arka的情况如何呢,这是特殊的.这是因为epistemic和deontic的法副词分别使用不同的单词.
Arka的法副词\footnote{作者自造词,``用于表达模态的副词''}虽然数量多,但没有多义性.
因此,仅凭这些无法判断是客观还是主观.


只有sen从“可以”的意思派生出“可能(klia)”和“可以(flen)”的意思.
“能(可能)”被分类为dynamic,而不是epistemic.而且“能”表示客观的能力,所以“能”→“可能”意味着从客观到主观的跨越.
另外,flen是deontic的一种,意味着从客观到客观的跨越.
因为法副词的意思从客观扩展到主观乃至客观,所以可以知道Arka的情态的根源在于客观.

更进一步说,在Arka中,epistemic与fal(deontic)等法副词不同,是用klia等游离副词来表示的,这一点也值得注意.
法副词相当于英语中的助动词,游离副词只是副词的一种,本质上和very、hardly等没有区别.
由此可见,在Arka,deontic属于法副词的语法范畴,而epistemic则被单纯地放在副词中.
也就是说,在Arka中deontic是比epismetic更受重视的模态,因此判定是客观的.

英语中的can有epistemic(可能)、deontic(可以)、dynamic(可以)三种用法,用法非常广泛.
can的原意相当于其中的dynamic,既然英语是客观把握的语言,模态的客观性就是dynamic > deontic > epistemic吧.
因此,Arka的sen(dynamic)扩展到klia(epistemic)和flen(deontic),表现了情态从客观向主观的扩展.因此,可以说Arka的情态的根源在于客观的把握.

\subsection{与格か间接目的語か/间接受身:客观}

阿鲁卡和英语一样,不存在利害关系.
与英语同为印欧语,拉丁语和西班牙语也有利害关系,例如西班牙语中“Me llovio”表示“我下雨了”,即“我被雨淋了”.
如果用阿鲁卡来形容的话,就是eskat (an) sin.表达``麻烦感''的模态以句末纯词的形式从外侧包住事件.

同样阿鲁卡不存在间接被动.间接被动主要可以分为影响被动(我被雨淋了(私は雨に降られた))和持主被动(我钱包被偷了(私は財布を盗まれた))两种.

关于“我被雨淋了”如上所述,使用句末纯词.不使用被动语态.an eskat yu虽然说得通,但不会给人添麻烦的感觉.

关于“我的钱包被偷了”,虽然像an eftat yu on gils sin那样使用被动语态,但``麻烦感''最终还是由sin来承担.

当然,即使没有sin,常识上也知道这是令人厌烦的内容.另外,也可以用``xe eftat an on gils"、"xe eftat gils ant"、"xe eftat gils it an''.
语气各不相同.

在最后一句中,被偷的钱包可能不是自己的,而是寄存的东西.
gils eftat yu it an这种被动语态的说法也是可能的,但是没有产生困扰感.
只是从常识上考虑,不加上sin也会给人造成困扰.当然,这里的句末也可以加上sin.
关于持有者被动,可以在字典里参考eft.

综上所述,判定是客观的.完全看不到日语的被动语态的用法.Arka的被动语态只在聚焦主动句宾语的时候使用.

\subsection{(英语的)中間構文:客观}

阿鲁卡原本就没有动词“自他\footnote{或者说及物和不及物}”,都是其他动词.主语sol在大部分情况下是agent,宾语yul在大部分情况下是object.

是极其明了的语言.

因此,像This book sells well(这本书卖得很好)这样的中间句法本身就不存在.

硬要翻译的话,应该是tu lei em atm ati di,但不自然.那么是el tau tu lei ati di吗?
不,与总称el相比,“大量购买者”具有具体性,所以是主语.结果,lan di tau tu lei是最自然的预测.
而且如果用阿鲁卡的语感来验证的话,确实是最自然的说法.如果把焦点放在书上,就写成tu lei tau yu ati di等.
与This book sells well相比,使用的单词和语法都发生了巨大的变化.翻译中间句法的时候要注意.
阿鲁卡虽然没有中间语法,但是其对应的译文是lan di tau tu lei这种sul语言的表达,所以判定是客观的.

\subsection{动词vs.卫星结构:主观}
表达移动的动词,在手段和样式的区别上,
可以分为多用动词本身来表达的动词框架语言(Verb-framed language)
和动词不变,用附属的前置词或接续词来区别的卫星框架语言(Satellite-framed language).

前者可以举出日语和罗曼语(法语和西班牙语).
例如在日语中,“进入(入る)”、“下降(下る)”、“通过(通る)”等作为动词而存在并经常使用.
后者包括英语、德语、俄语等多种印欧语,以及汉语等.
\footnote{像汉语的``走近科学",说"走近"是一个词,不如说一个以"走"为中心的偏正结构,这一点在"爬上栏杆''一词中体现更加明显.}
例如在英语中,使用“go in”、“go down”、“go through”等通用动词的说法.
德语也与此相似,使用加了词缀的分离动词.(来自维基百科动词框架语言和卫星框架语言)

关于这个,阿鲁卡是传统的动词框架语言.因为幼字比汉字更具有一词唯一的原则.作为客观把握的代表的ly, ls也保持着这种性质.
实际上从地球的语言来看,本来应该被分类为客观把握的法语和西班牙语也和日语和Arka一样,所以关于这个特征,把握的方式并没有太大影响.

\subsection{主观谓语:客观>主观}

用日语说“我很高兴”,却不能说“他很高兴”.如果不说“他看起来很高兴”,就不自然了.
这是因为日语是主观的把握.因为说到底是站在说话者的角度看对方,所以不知道对方是否真的高兴,这就是主观谓语.
像英语这种客观把握的情况,因为是从神的视角来看,所以没有主观谓语.

对于Arka来说,an na nau / lu na nau都是自然的.因此可以判定为客观.
但lu na nau in也是自然的.根本来说,虽然有客观的把握,但是特别是关系到感情和人际关系的时候主观把握,使用句末纯词的in, ter, yun等.
因为在这一点上掺杂了主观把握,所以判定是客观>主观.
\subsection{拟声词/拟态词:客观}

和法语一样,几乎为零.判定完全客观.
从历史上来看,除了tanta等几例拟态词之外,没有其他拟态词.

在现实生活中,异民族聚集在一起形成了Arka\footnote{见综述,Arka的创作者至少包含五种母语者},
所以在模棱两可的拟声词中,感觉无法共享,拟声词——尤其是拟态词——并不发达.
Kaldia也有很多异民族,因而拟声词不发达.
再进一步说,在现实中,梅尔参加(Arka的创作队伍)以后,产生了“这个声音象征这个意思”的所谓声音象征.
由于声音被特殊化为具体的声音象征,在暧昧的拟声词中使用的机会就更少了.
举个声音象征的例子.e表示与水有关的东西.现在也有er、eria、eri、lue等,不胜枚举.
在现实的Arka中,声音象征占据优势,所以拟声词更受冷遇.
顺便一提,Kaldia的声音象征在f、fv时期就已经发展起来了,这背后有ace理论这一神话学的理由.
因为是与语言学不同的领域,所以在此不做说明.请参照各自的词典.

另外,拟声词与拟态词不同,它非常丰富.特别是由于演绎音的存在,获得了比日语和韩语更有体系的拟声词.
为什么要将拟声词体系化到如此地步呢?因为如果不成系统,就很难在多民族之间产生共鸣.无论在现实中,还是Kaldia的Arbazard,都是如此.

\subsection{过去时叙述中的现在时:特殊}

假设现在是2011年,假设1991年发生的一年的故事.
在英语中,作为认知主体的读者从2011年开始客观地看待1991年的故事.因此,文章中使用过去时.
日语中虽然也有原则相同的用法,但也有时将认知主体埋没在故事中,使用现在时.
这是主观的看法,会增加临场感.在故事中使用“る”除了临场感以外还有其他用法,但这里重要的是日语通过移动视角主观地使用现在时这一事实.

阿鲁卡不属于这两者.从历史中截取这个故事存在的1991年.舍弃前后的时间,形成独立的时间.
因此,2011年存在的认知主体将被舍弃.也就是说,Arka的特征在于舍弃认知主体.
时态常常随着故事的进行用现在时来叙述.想象一下在电脑上看的视频,进度标显示视频当前的位置.
阿卡对故事的理解方式与动画相同,也就是以现在时讲述进度标的位置发生的事.
这和你几年在看那个视频没有任何关系,而且不管你是否在看画面,搜索器都在前进.
也就是说,舍弃认知主体,故事继续前进.因此,可以说它具有既不属于客观把握也不属于主观把握的特殊类型.
阿鲁卡就像一个空无一人的电影院.不管有没有观众都继续上映.

换句话说,在阿鲁卡,地句的默认时态是现在时.这不是无时制.表现出仿佛现在故事就在眼前进行着一样.
日语中默认的是过去式,所以会感到不协调吧.
确实,阿鲁卡的句子默认使用现在时,但从上下文的时间点来看,如果是过去的事件,就可以使用过去时.
也就是说,在谈及在进度标之前的时间点发生的事情时,在那里的叙述也可以使用过去时.

\subsection{直接/间接叙述:客观}
阿鲁卡的间接叙述非常发达.这一点和英语一样,判定是客观的.
但是,当主从句和从句的主语相同时,从句的主语要省略,或者换成nos.
换成nos的部分,会让人觉得从主句的主语的视点上看问题,不像英语那样客观地把握问题.

另外,关于时态的一致,阿鲁卡通过与主节的比较来进行.如果主句和从句的时间点相同,即使主句是过去时,从句也是现在时.

He said, “I am busy.”= lu kut ``an tur vokka''.
He said that he was busy. = lu kut nos tur vokka.

在阿鲁卡,间接叙述的频率远远高于直接叙述.
{\noindent} \rule[-10pt]{17.5cm}{0.05em}

\section{Arka是现代的认知样式吗?}
所谓“双重把握”,与偏向客观把握或主观把握中的某一方的方式相比,更接近人类本来的认知方式.
人们总是从主客两方面来把握事态,根据适当的情况选择适当的一方.

这和电视的播放方式很接近.电视的收录是用多台摄像机同时拍摄,适当地播放最容易传达给观众的摄像机的影像.
在足球比赛中,如果进球了,就会从球门的角度出发.在综艺节目中,如果有人想插嘴,就从环视全体嘉宾的摄像机切换到特写镜头,播放画面.
电视的影像接近于人的双重把握.因此,观众能够毫无违和感地接受电视画面.
如此一来,只把语言偏向于客观把握或主观把握的一方,从人类的认知模式来看,反而会感到不协调.

另外,从时间轴截取故事,在一个故事中经常使用现在时,就好像把故事放在一张DVD中,用拖动条表示现在的位置一样.
Arka故事中的现在时相当于与现实的时间轴分离的,DVD的拖动条.
无论是电视还是DVD的动画,总觉得Arka的双重把握和故事中的现在时等例子,Arka这个语言映射了现代人的认知模式.

\section{迟到到认知语言学考察}
我第一次意识到Arka的语言风格是在中学的时候.而强烈意识到的时候是高中时代.而想写本文是在制作``制Arka''的大学前半期.
但是直到2011年,我都没有将认知语言学的考察整理成一篇报道,只进行了片断性的考察.
Arka的情况是,在作者们接触语言学之前就有语言制作,所以经历了曲折.因此,为了不让后起的制作者重复,建议先学习语言学.

不过,自古以来就有一件不可思议的事.就像孩子自然地记住语言一样,
虽然没有特别商量,但在Arka用户之间,“这是一个很符合Arka风格的句子”这样的语感是不争的事实.
一开始当然不知道原因.但我想,既然在现实中有共同的语感,就一定有某种规则.我认为,只要进行语言学的考察,就一定能发现其中的规则.
虽然不知道那个规则是什么,但因为大家都掌握了规则,所以想着总有一天要进行总结考察,就把时间拖到了2011年.

现在找到了双重把握的认知方式,好几件事都发现“原来是这样啊!”.例如下述.
\section{语言相对论的再评价,以及杂感}
\subsection{认知主义和生成生成主义}
笔者以创造人工语言为目的开始了语言学.作为搞语言学的人,这是破例的进入方式.我一直从是否对Arka有用的角度来评价语言学的理论.
另外,在青春期接触了很多民族,对于不同民族对事物的理解是多么的不同,我也有深刻的体会.
\footnote{译注:我从网页上没有找到作者的名字,但是这真是付出了相当的思考和努力}

对于这样的笔者来说,最能引起共鸣的语言理论就是所谓的“萨丕尔·沃尔夫假说”.沃尔夫的言论可以在Whorf(1964)等中找到.
萨丕尔·沃尔夫的假说一般分为语言决定论和语言相对论.虽然笔者并不赞成决定论,但还是支持相当偏向决定论的相对论.
另外,从语言学的派系来看,笔者是明确的认知主义者.特别是对平克和乔姆斯基持怀疑态度.
其中,对于平克(1995年)的萨丕尔·沃尔夫假说的毫无论据的指责,我很难同意.
如果他在与不同民族的讨论中从零开始构建一种语言和世界------或者至少他在不同语言和不同文化方面有更大的造诣------
就能切实感受到语言对人类的巨大影响.维耶朱维茨卡(2009)对平克的批评一针见血.
另外,这本书也是想要了解语言和文化关联性的一本书.

笔者也并非一开始就反对生成主义.对于原本是理科的笔者来说,一开始反而觉得生成主义比萨比亚·沃夫的假说更适合——
至少从字面上看.但实际读过之后,才发现其理论与实际体验相差甚远,甚至觉得是不符合现实的理论.
\subsection{这种``不明白''是``无法理解''还是``无法产生共鸣''}

知道了这次的双重把握后,我曾想过“原来如此,所以啊”.
在日本看电视的时候,经常能看到人们讨论的样子.有时是政治相关的话题,内容各式各样,但经常能听到“不能理解”“不懂”的台词.
很少有人会说“理解但不共鸣”.这是为什么呢?

日语是主观把握.没有客观把握的倾向.逻辑适合客观把握,感情适合主观把握.
也就是说,日语在表达逻辑的时候也有主观把握的倾向.证据就是“分かりました(我明白了)”.
这句话的意思是“理解しました(理解了)”和“了承しました\footnote{了承(りょうしょう):理解;同意,晓得;谅解;了察,与``理解''相比偏向情感方面}
(明白了)”.逻辑和感情没有分化.
在客观把握的法语中,“Est-ce que vous comprenez?”(理解了吗?)和``d ' accord ? ''(明白了吗?)使用不同的单词是很自然的.
当然,在客观把握的法语中,也有entendre(理解)到Entendu(了解)的表达方式,所以不能统一表达,
Arka的loki(理解)中没有xiyu(理解)、yuta(接受)、okna(认同)的意思.即使确认过语料库,要说“知道了”的时候,也会回答“xiyu”而不是“loki”.
但是,即使是Arka,在以服从对方为前提的环境中,仅仅是理解了,实际上就意味着“同意了”.

虽然不是说英法都做得很好,但日语给人的印象是不能把逻辑理解和感情理解分开.
也许日本人认为说了“わかった(明白了)”就等于接受了对方的要求,所以不会那么轻易地说“わかった(明白了)”.
结果,当讨论陷入胶着时,“わからん(我不懂)”“まったく理解できん(完全不能理解)”等台词就会乱飞.
恐怕他觉得一旦说了“わかった(我知道了)”,就会被对方抓住把柄.即使只是表示“理解した(理解了)”的“わかった(明白了)”,
也有被逼问“你那个时候不是说明白了吗?”的危险.
但是在Arka里,即使说an lokik ti(理解了你),也不意味着有同感,所以不会在事后做出敷衍的说法.
\subsection{Arbazard人说:``理解但不共鸣.''}
关于这件事Arbazard人怎么样?现实中的古Arka也是如此,
Arbazard人的典型说法是“啊,原来如此,我明白你想说什么,但我不那么认为”.
这是将逻辑和感情完全分开的表现.
日本人不怎么这么说.实际上,如果这样说的话,恐怕会造成杀气腾腾的气氛.
但是Arbazard人非常喜欢这种说法.这是双重把握的结果吧.
中、日、韩、英、法等国家都倾向于“主客”中的一方,没有同时考虑“主客”的习惯.
“理解但不共鸣”的说法在日常生活中很常见,如果这种习惯是双重把握造成的,那么语言对思考的影响就不能轻视了.

Arbazard这个国家原本就受到了Lutia的魔法兵团、Metio的魔兽兵团和Artia的武士的威胁.
这样的Arbazard之所以能成为世界最强的国家,是因为他总是合理地思考,不搞枪打出头鸟\footnote{原文为``出る杭は打たれる'',是类似的俗语}那一套,
赞扬伟人而不诽谤伟人,惩恶扬善,不拘泥于传统而坦率地吸取好的东西.
如果没有这种国民性,这个土地平坦、地势平坦的国家就不可能一直如此强大,甚至生存都难以为继.

Arbazard人的谈判方式非常随意.说:“我到这里妥协.你到哪里妥协?这就是妥协点了,妥协不了就打啊.
但是我的战斗力是这.你的战斗力是这.推测死者的人数是这.来吧你?”的态度,完全只考虑合理性.
另一方面,“战争的结论以上就讨论出来了,从战斗力说确实我们我胜利.但是从情面上讲,彼此不愿出现死伤吧,多悲哀啊,
于是请你多少让步些,仔细考虑要不要再一次发动战争吧”这样的说法.

对他们来说,首先要有逻辑,然后附加感情方面.完全把逻辑和感情分开,在日本是不太普遍的看法.
在没有这个习惯的人看来,可能会觉得“太机械了,太冷漠了”.相反,你可能会认为“理性、有逻辑、聪明的人,自然会发展”.
\subsection{语言的影响仅限于思考的习惯}
我经常感到Arbazard人用主观和客观同时把握事态的习惯很独特.
不仅是日本人,就连美国人和法国人也不太能看到.
但在这次考察中,我了解到这是由双重把握造成的,
于是我重新评价了一直以来反响强烈的语言相对论.

但是,即便如此,也没有到支持语言决定论的程度.
语言对思考习惯的影响是毋庸置疑的,但它并没有决定思考的力量.
例如,笔者在做Arka之前就已经把主客分开考虑了,原本就有双重把握的习惯吧.
作为日本人,我认为这是很少见的,如果语言决定论是正确的,
那么笔者学会主客分离思考应该是在学习Arka之后.

\subsection{角色语与语言相对论}
另外,语言相对论与金水(2003,2007)所倡导的角色语也有关联.

我们来比较一下拥有丰富的角色句尾的日韩和没有的英法吧.
后者的说话者能同样感受到前者所感受到的性格吗?不论怎么说,至少关于角色句尾传达的角色性确实遗漏了.
从具有丰富第第一人称代词的语言翻译成不具有丰富第第一人称代词的语言时,也会出现这个问题.
举个例子吧.“わたくし、磯鷲早矢と申しますの”和“拙者、緋村剣心でござる”如果翻译成英语,
都只能是“I am Haya”或“I am Kenshin”之类的说法.

日语中有“わたくし\~{}ですわ”这样让人想起大小姐角色的词语.
日本人有根据第一人称和终助词的使用方法来观察说话人个性的思维习惯.
当然,即使是没有终助词或只有一种第一人称的语言,也可以通过其他词类和措辞来判断对方的个性.
但是,仍然没有产生用第一人称或终助词来观察对方个性的思考习惯.
从这个意义上来说,语言确实会对思考产生影响,由此可以感觉到语言相对论和作用语之间存在着关联性.

Arka和日语一样,人称代词等丰富,擅长表现人物性格.
因此,可以说Arbazard人培养了用那样的要素推测对方的人性的思考的习惯.
\section{参考文献}
Langacker, R. W.(1985) ``Observations and Speculations on Subjectivity'' 
Haiman (ed) ``Iconicity in Syntax'' pp109-150

John Benjamins Publishing Comapany――(2008) ``The relevance of Cognitive Grammar for language pedagogy'' 
Sabine De Knop (ed) "Cognitive Approaches to Pedagogical Grammar: A Volume in Honour of Rene Dirven 
(Applications of Cognitive Linguistics)" p18 

Mouton De Gruyter Benjamin Lee Whorf(1964)``Language, Thought, and Reality: Selected Writings of Benjamin Lee Whorf'' 
John B. Carroll (ed) The MIT Press

アンナ ヴィエルジュビツカ(2009)『キーワードによる異文化理解』 p22 而立書房

スティーブン ピンカー(1995)『言語を生みだす本能〈上〉』日本放送出版協会

庵功雄(2001)『新しい日本語学入門―ことばのしくみを考える』スリーエーネットワーク

池上嘉彦(2004)「言語における〈主观性〉と〈主观性〉の言語的指標 (1)」『认知言語学論考』No.3 pp1-49 ひつじ書房――(2005)「言語における〈主观性〉と〈主观性〉の言語的指標 (2)」『认知言語学論考』No.4 pp1-60 ひつじ書房

金水敏(2003)『ヴァーチャル日本語 役割語の謎』岩波書店――(2007)『役割語研究の地平』くろしお出版

河原清志(2009)「英日語双方向の訳出行為におけるシフトの分析―认知言語類型論からの試論―」日本通訳翻訳学会\subsubsection{翻訳研究分科会(編)『翻訳研究への招待』第3号 pp29-49}

中村芳久(2004) 「主观性の言語学:主观性と文法構造\subsubsection{構文」 中村芳久(編)『认知文法論Ⅱ』 pp3-51 大修館書店}

森山新(2007)「认知言語学的観点による日本語の連体修飾研究-連体修飾節\subsubsection{ノを用いた連体修飾を中心に-」日本学報 72. pp41-58――(2009)「日本語の言語類型論的特徴がモダリティに及ぼす影響 :グローバル時代に求められる総合的日本語教育のために」比較日本学教育研究センター研究年報, 5, pp147-153}
%``([a-z]+)''
%``$1''

\chapter[语调]{语调}
\emoji{l_asex_kal}
既然讲完了语法,今天我带了朋友来哦.

\emoji{a_nax}
soonon♪
听说紫苑在学Arka,就过来玩了哦\~{}


\emoji{x_nau}
啊,这不是占卜师Alia嘛,欢迎欢迎.

\emoji{x_nik}
诶,等会!

为什么你也在说汉语啊( ゚Д゚)


\emoji{a_reps}
啊哈,我一切皆有可能哦。占卜师什么都能看见哟.

你看,拿着这个水晶,不管是考试的答案,还是彩票的中奖号码(呼呼呼...


\emoji{x_seernik}
诶呀诶呀,这已经超过占卜师的领域了吧 w

\emoji{x_dagd2}
我记着《紫苑の书》的一个优点是就是排除了这种投机取巧吧?

别老整这胡来胡来的好吧.我作为主人公可不能坐视不理啊.


\emoji{a_niit}
haan, xom xenon xook fan al axel kokko arka hot, nee lein, nonno?

(OK,我们现在全篇用Arka交流吧,呐,lein?)

saa, noel jins vil toal iskan rok sarlots t'inar atu e(axte

(不知道有多少读者会点右上小红叉w)


\emoji{x_vexl}
――别别别这么我求你了!!(゚ロ゚;)


\emoji{l_sena}

好好好.

昭和漫才还是打住吧,今天是关于语调的。


\emoji{x_lek}
额!!Σ(°Д°;


\emoji{l_asex_kal}
你觉得「Arbazard」哪里重读比较好呢?

其实arbazard在哪里重读都可以.


\emoji{x_lo}
这就是说不用记单词重读的位置?比英语简单啊。

但是,因为不知道哪里可以放重音,反而更迷惑了吧.

有什么决定方法么?


\emoji{l_diina}
每个方言都有自己的放重音的方式。
Arbazard有3个方言,我和Alia说的北方方言和日语重音相近.


\emoji{a_rans}
嘛,虽说是日语,但也有各种各样的方言。

这里说的是东京方言的外来语口音。


\emoji{x_niit}
原来如此。那么“Arbazard”在Lein和Alia的发音中“za”的部分也很强吗?


\emoji{l_niit}
对就像读片假名一样\footnote{意思就是有些是开头重音,有些是开头低,中间高,后面再低(主要是在复合词中的后一个词下去).}

其实重音放错了也没事,不必在意.


\emoji{a_niit}
重音法则虽然有,但很难,这里就不讨论了。

法则也有例外,不过,后面介绍的词典里有“重音”的标签,所以没关系。

\chapter[疑问句,否定句]{疑问句,否定句}
%\chapter[短标题显示在页面]{长标题显示在目录}

  



\emoji{l_diina}
这是我前几天在公园散步时看见的花.

紫苑,这是樱花吗?
\begin{figure}[H]
\includegraphics[width=0.7\textwidth]{ARKA/ping.jpg}
\end{figure}

\emoji{x_lo}
额,这是梅花,不是樱花哟

呐,在Arka里怎么说``这是梅花.''?

\emoji{l_asex_kal}
``这是梅花''应该说``tu et ping''。``tu et''就是上次讲过的

``这是樱花''也差不多,说``tu et seron''。

``这是樱花吗''就在句子最后加上mia.``tu et seron mia?''

不过很多场合我们不特意加``mia'',只是在句子后面加一个问号.


\emoji{x_pil}
疑问的时候语调要上扬一些呢.
相比于英语的``Do you...?''Arka要简单呢。跟汉语在句末加``吗''比较像.


\emoji{l_sena}
另一方面,``这不是樱花''应该说``tu de seron''.
de一个词就表示系动词的否定,像是``isn't''一样.
顺便一说,女生不用``de'',而用``te''.


\emoji{x_niit}
那么我其实应该说``tu te seron''吗.

%ともあれ、「~でない」はdeね。「~DEない」と覚えようw ------翻译不过来的梗
另外,``不写''应该怎么说呢?


\emoji{l_xanxa}
be动词以外嘛,就在动词前面放副词``en''.

axt是``写''、那么``en axt''就是``不写''.


\emoji{x_demo}
哦......``紫苑不写Arka''就是``xion en axt arka''吧。

让我总结一下:

\textbf{
疑问句:只是在句尾加``mia''\\
否定句:be换成de,另外就是在动词前面加``en''.
}

\emoji{l_deyu}
最后来做个小测试吧.

试着写下``这是猫吗?'',``这不是猫,是狗.'',``紫苑会写Arka吗?''这三个句子
答案就在下回的开头.atte!(加油!)




%``([a-z]+)''
%``$1''



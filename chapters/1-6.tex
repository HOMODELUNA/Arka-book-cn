\chapter[系动词]{系动词}
%\chapter[短标题显示在页面]{长标题显示在目录}

  

\emoji{x_sena} 
呐, Lein,这是啥?

\begin{figure}[H]
\includegraphics[width=0.7\textwidth]{ARKA/ket.jpg}
\end{figure}

\emoji{l_ket} 
猫!

喵喵\~{}.

我之前给你说,猫在Arka里叫做``ket''.

``这是猫'',就说``tu et ket.''


\emoji{x_niit} 
Arka里注释在最前面呢,所以``tu''就是``这个,'' ``et''意思是``是.''


\emoji{l_asex_kal} 
对,``et''是系词.``tu et oma''意思是``这是一只狗.''


\emoji{x_lo} 
``oma''的地方可以填形容词吗?


\emoji{l_sena_deyu} 
可以.``tu et kai''意思是``它很大.''


\emoji{x_loki} 
变过去式就加``-at'',对吧?

那么``它之前很大''就是``tu etat kai''吗?


\emoji{l_pels} 
不是的,应该说``tu at kai.''

系词出现得相当频繁,所以我们不说``at''而是说``etat.''

这对``or''(进行时)和``ik''(完成时)也适用.

``tu or kai''意思是``它正在长大,''而``tu ik kai''意思是``这已经长大了.''


\emoji{x_tisse} 
哈哈,这样我也不用记``was''或者``been''这类的分词了.

Arka的系动词就是``et.'' ``tu et miik''意思是``这是一个苹果.''

句子的顺序就和``xion axt arka'' (紫苑写Arka)一样呢.记得住记得住.

顺带一说,``紫苑不写Arka''和``紫苑写Arka吗?''应该怎么说?


\emoji{l_nax} 
你是说疑问句和否定句吗?

OK,我下次就教你.



%``([a-z]+)''
%``$1''



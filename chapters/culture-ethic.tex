\chapter[人工世界的伦理观]{人工世界的伦理观}
在创作人工世界的时候不应该被自己的伦理观左右.

举个例子,奴隶制听起来印象就不好,但是如果没有奴隶制,希腊哲学就没有产生的土壤.
正是底层人包揽了杂事,使智者得以研究和思考,学问和技术才得以发展.
像共产主义一样平等地生活,对国内也许不错,但如果受到强敌入侵,就难逃灭国和殖民的命运.

Lestir这样的强国当然会有奴隶制吧.
但即使如此,Arbazard也不能说是邪恶的国家.

不,认为这是一个邪恶的国家是很随意的,但我并不是按照现代日本的伦理来创造世界,那样就完全不现实了.
要创作出真实感,奴隶也好,强奸也罢,这些是自然产生的东西,就必须有设定.

即使从伦理上讲没有为好,但如果要创造世界历史,那么屠杀、战争、抢劫、杀人、拷问都应该有,必须要创造.
在创作人工世界的时候不应该被自己的伦理观左右,而应该保持客观和严肃的态度.


同时,伴随着社会的发展,``委婉''必然被创造出来.
现代日本可以说是没有奴隶了吧,可时薪不到数百元的劳务派遣和黑心企业的正社员,和奴隶什么区别吗.
薄给激务中消磨了人生,落得以过劳死收场的话,也许当奴隶会更好吧.
\footnote{译者是中国人,大部分读者也是.这种现象要么我们见过,要么我们就正在经历.
不论怎么样,我希望大家能够珍惜自己的身体.如果可以的话,最好能帮助一下同样受苦的人.

一点私话:从物理的劳累来说奴隶并不比黑心企业轻松,但社会发展至今,总是有些变化的地方.给底层施舍一缕希望,这就是最美丽最残酷的行为吧.简直是艺术.
}
%中岛美雪阿姨的一句歌词,通往幸福的路有两条,一条是实现所有的愿望,一条是舍弃所有的愿望.

Arbazard在底层也有收入很低的奴隶.
但是高度发达的社会倾向于委婉地表示,毕竟还要顾及对外形象.
所以我必须换个说法,不说奴隶这个词,而称之为派遣员工.其中不能夹杂作者的伦理观.
我想,``虽然称呼不同但实质上还是奴隶,Arka不需要这种掩人耳目的词''.
但是又想到高度发达的社会当然需要一些``委婉''的措辞、还是觉得有必要造.

调查屠杀和拷问的历史,创造设定的同时,也要创造‘现在去见你’那样的感动故事.

人确实是善恶混杂的生物,既有邪恶的一面,也有神圣的一面.
\chapter{死生动词\&后记}
\emoji{l_sena_deyu} 
这就是Lein课程1的最后一节了.

今天我来教你死生动词.\footnote{英语原文是switching verb,说这对动词就像开关一样.}


\emoji{x_seernik} 
好可怕的名字!


\emoji{l_nakx} 
哎嘿嘿(\^{}-\^{};
生动词"ar"意思是"激活某物,"死动词"is"是"关闭某物."
就像汉语里的"开灯"和"关灯"一样

在Arka,"ar"和"is"这对动词的职责要大一些:"ar ata"是"开会\footnote{汉语里就没有"关会"这种说法,而是单独有"散会","闭幕"之类的动词.},
"而"ar pam"就是"开灯."


\emoji{x_tisse} 
看起来非常有用呢.
"ar +设备"就是"激活设备." "ar + 事件"就是"举行事件."

\emoji{a_rans} 
yeah.所以我们叫"ar"生动词.

"is ata"意思是"结束会议,"而"is pam"就是"关灯." "is"就叫做死动词.

你不需要记很多搭配,只要用"ar"和"is"就能表达很多复杂的句子.


\emoji{x_pilp} 
明白了

所以"alia ar lein"......


\emoji{l_vexl} 
我可不是道具!(>\_{}<)


\emoji{x_naki} 
抱歉啦 :)


\emoji{a_pilp} 
(Lein说是道具的话还真是方便......全自动做饭洗衣机w)

    
\emoji{x_nal} 
这样Arka讲座的理论编就告一段落了吧?

二位也辛苦了.多亏你们的教导,我现在对Arka掌握了许多。



\emoji{l_nax} 
你也累了吧,紫苑.

也感谢各位读者们 :)

理论编到此结束\~{}\~{}

我下次在Lein课2会给你讲一个Arka爱好者画的漫画,无论如何都来一下哦\~{}
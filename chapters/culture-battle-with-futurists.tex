\chapter{与未来人的战争}
先验地从零开始创造人工语言和人工世界,
是人工语言制作中难度最高的领域.

到2012年为止,地球上几乎没有人在实践这个领域,Arka规模的人则完全没有.
Arka和卡迪亚是时代的奇点,是时代的转折点.
本来在这个时代存在还为时过早.应该在计算机和互联网进一步发展之后才会诞生.

Arka之所以抢先一步开创了人工语言的历史,是因为作者着眼于未来.
坦率地说,作者根本不把现代人当回事.战斗的对手是未来的自己.
为什么要这么做呢,原因很简单,首先是因为现代没有Arka的对手语言.
另外一个理由是,考虑到将阿尔卡留在人工语言的历史中.

现代没有能和Arka战斗的对手.因为现代人从一开始就不理睬.
我之所以对其他人工语言和人工语言制作者不感兴趣,是因为根本没有可以成为我竞争对手的人才.
如果有竞争对手的话,只能是在技术进步的未来.所以常常只会意识到在未来人看来,现在的自己是怎样的.这种感觉一般人很难理解.

我一直在思考如何与未来人战斗.这类似于如何用旧刀与新枪作战的问题.
人工语言的制作,大致分为语言自身、词典、语料库等一手数据的制作,以及小说、插画、漫画、游戏、网站等吸引客户的二手数据的制作.
后者的工作——至少对我来说——是为了让现代人了解自己的语言,从而在未来将自己的语言连接起来.
说到底,后者不过是为了登上与未来人战斗的舞台而进行的作业,语言本身的构建只能是前者.

语言作者的工作终究是前者.后者只不过是一种宣传.
因此,作者应该尽可能地把有限的精力用于前者.
但可悲的是,越是能把后者委托给他人,越不会出现最初的理解者和赞同者.
因此,作者必须具备全方位创作内容的能力.
语言以外,小说、图画、漫画、歌曲等,都需要有相应的创作能力.必须非常灵活.

接下来是后来者的内容创作,在这一点上,未来人比我们优秀.
例如在2012年,如果个人委托制作漫画的话,每本书需要100万左右的金额.
在不外包的情况下,有必要自己掌握漫画这一技术.而且还要花费几年的时间来完成这项工作,承担停止语言制作的风险.
通过网络将漫画委托给半专业人士的话,页面单价在5000 \~{} 10000日元左右.即使是5000日元,200p就是100万日元.

页面单价的高低大部分是人工费用.
写漫画必须先写名字,打草稿,画线条,贴木条和色调,画背景,在喷模上印图案.
据我认识的漫画家说,一页要花6个小时左右.

但是,10 \~{} 20年之后,情况就会完全改变.
用3d数据制作角色和背景素材等,进行成像、配置、拍摄.
一旦建立了模型,无论从哪个角度、哪个角度都能简单地画出来.
用这个方法可以比以前更有效率地制作漫画.而且是彩色的.
最重要的是,不会画画的人也可以制作漫画.

这个构想是在2000年兴起的,我以为将来会发展成这样,而comi PO软件的出现,证明了我的预想是正确的.
恐怕在今后的10 \~{} 20年里,这种采用3d建模的彩色漫画将会兴起.一般认为以往的漫画制作方法会发生变化.
这样一来,黑白2d漫画就会和现在的黑白电影一样过时.

也就是说,未来的人工语言制作者可以比现在更简单、全彩地制作漫画.
假设现在我们花了几年,几百万来制作漫画.而未来人却能在短时间内完成.
这样一来,在内容制作方面与未来人竞争,绝对没有胜算.
在未来人看来,现代的内容终究是过时的.
现在制作内容作为让现代人知道自己的语言的方法是有益的,向现代人普及语言对于将未来人引向自己的语言这一点上是有益的.
但这些内容无法吸引未来人.说到底,它只不过是为了让未来人接过接力棒而吸引现代人的工具而已.

话说回来,在内容制作上与未来人对抗也没有胜算.
那么,我们应该如何面对未来人呢?
那就是在即使是未来人也绝对无法机械化、自动化的领域中交锋.

未来一定会有好的翻译机.在遥远的未来,如果创造出自己的语言,相信会比现在更容易转换成各国语言.
内容也很容易制作.不仅仅是漫画.动画也好,游戏也好,都变得容易制作了吧.
但是词典、语料库、小说、世界设定的制作都必须由人类自己来完成.
将来甚至连文件都将由机器利用庞大的语料库和模式分析自动完成.
但仅限于定型的文件.从零开始制作异世界和异语言这种非常规且有创造性的文件,机器是做不到的.
无论经过几百年,这都是人类要做的工作.

也就是说,如果想从零开始创造异世界和异语言,那么原始数据最终还是得由人类来创造.
这是我们原始人与未来人对抗的唯一线索.
说到底,内容只不过是吸引他人的工具,原始数据才是语言和世界本身构建的尺度.
即使是未来人,也不能偷懒.也就是说,语言和世界的构建是任何人都无法忽视的.

正因为如此,我们才应该制作第一手数据.
最终,原始数据反映了制作的程度,只有这部分无法实现自动化,所以最重要的部分永远无法实现自动化.
不过最重要的部分始终无法实现自动化倒也是一种幸运.也就是说,无论是未来人还是我们,核心的部分都是平等的.

当然,未来人将使用比现在更简单、更方便的输入设备.
通过语音输入制作文件,可以在不伤腰和肩的情况下完成大量的文件.
通过网络得到的信息也会变得更加丰富和便利,造词时进行的概念调查也会比现在更容易进行吧.
从这个意义上说,一手数据的制作也还是对未来人有利.

虽然有利,但绝对无法实现自动化.正因为如此,我们只能在核的部分战斗.

我不把现代人放在眼里.为了与未来人对抗,制作第一手数据.在那里赌上了信念和生命.我就是这样活过来的,今后也会这样吧.
这是自耶夫达以来,第一次将人工语言这种无法赚钱的东西创作到如此地步,在艺术语言方面,这是继托尔金之后的第一次.
就连托尔金也没有以这样的精度和数量来创造世界和语言.

自己对抗的是未来的自己.准确地说,是像未来某个时间点会崛起的自己一样的人造语言家.
如果自己是未来人,抱着同样的热情创造人工语言会怎么样呢?要想和自己交锋,在哪里一决胜负呢?
我就这么想着,工作了好几年.答案就是制作庞大的一手数据.


有时在内容制作上,漫画是最有效的.
声音、小说、漫画、动画、游戏等都可以作为吸引他人的内容.
其中最容易制作的是小说,但小说全是文字,很难读.门槛很高.
因为声音要求用户有听力能力,所以门槛很高.
游戏虽然可以玩,但是在语言的学习和学习方面,玩会让人产生意识,所以效率不高.

那么,动画和漫画哪一个更优秀呢?
在人工语言的普及这一点上,压倒性的是漫画.
动画片首先要求观众有听力能力,这一点门槛就很高.
就算加上字幕也解决不了问题.因为字幕会在一定的时间内自动播放.
初学者和熟练者阅读文章的速度完全不同.尽管如此,字幕还是以一定速度播放.

漫画的话,读者可以以自己的速度阅读.也不要求听力能力.在这一点上,比动画在语言的普及上更胜一筹.
不过,动画有唯一的优点.那就是权威.
制作动画比制作漫画更费力气.宣传自己的语言也有原创的动画,这样更有权威性.
不过,那也不过是最近几十年的事情.到了个人也能简单制作动画的时代,作为权威的动画也就失去了意义.
这样一来,单纯的漫画所具有的便利性得到了更高的评价.
因此,综合考虑作为普及工具来看,漫画可以说是最优秀的媒介.

然而我们应该首先进行漫画的制作.恐怕一开始即使花费劳力或金钱也应该用2d制作吧.
而且,当个人也能简单地用3d建模制作漫画时,就应该用3d全彩制作.
制作内容虽然只是吸引顾客,但作为与未来人战斗的入场券是必要的.
无论如何,如果未来人不懂自己的语言,那就连战斗都做不到.
\chapter{如何使用字典}
    

\emoji{l_nax} 
大家好,我是Lein.

我在初Arka1里面已经讲了Arka的语法.

不过我觉得大家会想"我虽然学习了语法,但是仍然不会读Arka."

紫苑,你呢?


\emoji{x_niit} 
确实,我知道Arka是SVO,形容词在名词后面,但是仍然不会读.

\emoji{l_sena} 
所以,我们在Lein课2里面就教大家如何通过依旧学了的语法阅读Arka.(*°---°)


\emoji{x_nakx} 
说真的,背单词太难了.

确实Arka的语法比英语简单,但是怎么也逃不掉背单词啊.


\emoji{l_ket} 
碰到不会的词就查Arka的在线字典好了.

你不用在电脑上安装什么.并且信息总是更新的,靠得住. 


\emoji{a_nax} 
这是一个电子词典,比纸质的词典更方便.

只要查的次数足够多,那么不用费力记单词也能记住.

首先,你可以查\href{http://conlinguistics.org/klel/}{Diaklel},Arka的在线词典.

\emoji{x_sena} 
Arka的词典被叫做"幻日词典"呢,"幻"是指Arka么.\footnote{英语原文称"Lunar Dictionary"}

在左上的检索栏输入么?


\emoji{l_sena_deyu} 
用普通的键盘就行了.

你也可以按拉丁字母序查找单词,而不用Arka的字母序.


\emoji{x_demo} 

搜索结果按字母序列出.

我以为它会像Arka字母序一样列出来呢,像 t, k, x, s...


\emoji{l_sena} 

Arbazard的字典里,单词以Arka字母序排列.

不过在Diaklel,有一段时间单词是以Arka字母序排列的,但是对用户不友好所以在08年就停止了.


\emoji{a_lo} 

Diaklel在2005年以Arka字母序编写,但学习者用起来非常困难.

不过现在非常趁手就是了.


\emoji{x_lo} 
Hmmm.另外,它由17000多个词.对于人造语言来说真的很大.

我的名字也在里面吗?哦,字典里没有"xion".因为我是地球的女生么?


\emoji{a_naki} 
不一定.也许你应该查一下你的全名,"xion hazki".

或者点一下"模糊检索(あいまい検索)"?

模糊检索会找出任何包含"xion"的词或短语.


\emoji{x_lo} 
这样啊

诶,我找到了.

说"她是紫苑之书的女主角,一名日本的高中生,故事从她的十七岁生日开始. 恶魔Meltia将她召唤到Atolas.
在这里她见到了另一位女主角Lein,,和男主角Arsye.她和恶魔之杖Varde战斗,拯救了Arbazard."

看起来字典总结了书中的内容呢.


\emoji{l_diina} 
Diaklel不只是字典,更是一本百科全书,对文化也有好多的记载.
关于文化的记录都写在[culture]一栏.查查看.


\emoji{x_knoos} 
Lein,苹果在Arka叫什么? 我想通过"苹果\footnote{其实要用日语词"リンゴ".}"这个词查Arka的对应词.

难道这个字典如果不知道Arka词就不能用了么?


\emoji{l_deyu} 
可以用.把"标题栏检索(見出し語検索)"换成"译语检索(訳語検索)."

你现在就能把它当成日-Arka词典来用.

去掉"模糊检索"然后把"リンゴ"写到搜索栏里.按回车或者查询键.


\emoji{x_niit} 
找到了,Arka里苹果是miit.

你在"译语检索"里也能用"模糊检索"吗?

    
\emoji{a_niit} 

可以.

你也可以查"青苹果(青りんご)."里面的"苹果".


\emoji{x_nau} 
哦,太好了(°∀°)

这个字典确实有很低功能呢.


\emoji{l_uni} 
\href{http://conlinguistics.org/arka/study_yulf_140.html}{字典的巧妙使用方法}中还有很多其他的诀窍,请读一读
那么下次使用aav一项,说明本文所讲的用法。
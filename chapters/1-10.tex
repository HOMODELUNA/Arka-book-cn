\chapter[格词]{格词(前置词)}
%\chapter[短标题显示在页面]{长标题显示在目录}

\emoji{l_xanxa} 
好,我们之前搞定了难受的代词,现在来学习Arka的格词\footnote{英语为 Preposition (一般称介词),不过她还特地造了一个词 caser.}吧.


\emoji{x_nau} 
Lein老师,格词是什么呀?


\emoji{l_niit} 
这是Arka的一个术语,就像英语里的介词一样.

举个例子,``to''是``a,''而``from''是``i.''


\emoji{x_asex} 
我们已经学了介词``a.''
``Lein给紫苑一个苹果''就是``lein fit miik a xion.''
``a'' indicates a destination, a recipient and so on.


\emoji{l_asex_kal} 
Yeah.英语里介词数量很小,但是Arka的介词,规模可是,相当宏大.


\emoji{x_knoos} 
二十几个?


\emoji{l_txu} 
额...100个以上...(汗)


\emoji{x_lek} 
我缓一会\quad w.


\emoji{l_myulla} 
嘛,其实常用的格词也就是十多个.

再说,大多数格词其实是和名词差不多的东西.

``hal''是``在上方,''但是也可以作名词.


\emoji{x_pil} 
是这呀,你不早说,吓死我了.

英语里``Front''是个名词,但``in front of''是介词短语.但对于Arka,它们的形式都是``sa.''你不需要单独记它.


\emoji{l_xakl} 
说回来,有这么多介词有啥好处呢?


\emoji{x_fron} 
唔?

我想想... 唔...想不来.


\emoji{l_diina} 
反过来想.

像英语那样前置词太少,不就产生了各种问题嘛.


\emoji{x_lo} 
要是没有很多介词,就要把一个介词赋予多重意义.事实上,``on''和``in''就有一大堆意思.

英语学习者就经常纠结什么时候该用什么介词.


\emoji{l_niit} 
相反,感谢这丰富的数量,Arka的学习者就清楚地知道哪里用什么介词.

确实你要记``frem''(附近)or ``fol''(...期间)之类一大堆介词,但它们仅仅是普通的形容词和介词,费不了多少功夫.
我把常用介词列成表:
\begin{table}[H]
    \begin{tabular}{|c|c|} % {l|c|r}<-- Alignments: 1st column left, 2nd middle and 3rd right, with vertical lines in between
    \hline
	a&  	为,到 \\\hline
  	i&  	从 \\\hline
  	kon&  	使用 \\\hline
 	ok&  	和...一起 \\\hline
	ka&  	在(地点) \\\hline
	im&  	在(时间) \\\hline
	man&  	因为 \\\hline
	ol&  	如果 \\\hline
	\end{tabular}
\end{table}

  
    

\emoji{x_loki} 
``man''(因为)和``ol''(如果)在Arka里不是连词而是介词吗?

``Shion和Lein一起写Arka''就是``xion axt arka ok lein''?


\emoji{l_uni} 
bingo!

别把``ok''说成``kon''啊,我可不是笔啊.



%``([a-z]+)''
%``$1''



\chapter[副词]{副词}
%\chapter[短标题显示在页面]{长标题显示在目录}

\emoji{l_deyu} 
\hypertarget{chapter-adverb}{我们}上次学习了Arka的时态和体态,紫苑,你还记得吗?


\emoji{x_vesn} 
动词加``-at''变成过去时.对应的,``-or''变成进行形,``-ik''变成完成形.

``写''是``axt,''那么``axtat''就是``已经写了,''而``axtor''意思是``正在写'',``axtik''意思是``写完了.''

英语的进行时和完成时还要加``be''或者``have''之类的,但Arka里面加后缀就行了.

那未来怎么表达呢?像``将要写.''


\emoji{l_niit} 
你应该在动词前面加一个副词, ``sil.''

``Axt sil''意思是``将要写.''跟``-at''不一样,``sil''不是后缀,所以不要写成``axtsil,''而应该写``axt sil.''


\emoji{x_niit} 
``Axtat'' (写过了)比``axt sil'' (将要写)要短呢.

这是为什么呢?


\emoji{l_sena_deyu} 
好问题!因为过去时出现得更加频繁.

过去时,进行时,和完成时比较短是因为它们出现的更频繁.不常出现的时态就用``sil''之类的副词表达.

这些是Arka的其他副词.我把常用的副词打成表:
\begin{table}[h!]

    %\caption{常用副词}
    \begin{tabular}{|c|c|c|} % {l|c|r}<-- Alignments: 1st column left, 2nd middle and 3rd right, with vertical lines in between
      \hline
	  \textbf{词汇} & \textbf{简要翻译} & \textbf{意义}\\
      \hline
      lax&  想要做&  希望\\\hline
  	van&  要做&  意志\\\hline
  	sen&  能&  可能性\\\hline
  	vil&  不能&  没有可能性\\\hline
  	das&  为什么不&  提议\\\hline
  	fal&  必须&  义务\\\hline
  	flen&  可以&  许可\\\hline
 	xiit&  让我们&  劝诱\\\hline
  	yu&  被&  被动\\\hline
    \end{tabular}

\end{table}    

\emoji{x_asex} 
Arka的副词相当便利呢.像英语之类的,助动词先得一堆.

当你表达``想要'',只需要加上``lax'',``为什么不''只要用``das''时,那可舒服了.

另说一下,这些都能是动词吗.我是说,一般的动词都像``高高地''或者``猛烈地'',这样.


\emoji{l_sena} 
``强,猛''是 ``vien'',``猛烈地''是 ``vienel.''要把形容词变成动词,你只需要加``-el''后缀就行了.

Arka有两种副词,一种想辅助词一样的,和形容词加``-el''转化来的.

结尾是元音的,就不加``-el'',而是加``-l''.``aalo''意思是 ``灵巧,''那么``灵巧地''就是``aalol,''而不是``aaloel.''

来个小练习,``紫苑可以写Arka''要怎么说?
%dexterous:灵巧

\emoji{x_lo} 
``xion axt sen arka''?


\emoji{l_ket} 
嗯~,对了!♪


\emoji{x_knoos} 
嘿, 我在表里找到了``被动语态''呢.这是啥?


\emoji{l_diina} 
``yu''产生被动语态. ``xion axt arka''意思是``紫苑写Arka,''这是主动语态.

要把它变成被动语态,就--\\
\textbf{
\quad1) 在动词后面加``yu'': xion axt yu arka\\
\quad2) 把主语和宾语倒过来: arka axt yu xion}

--就这样.

``arka axt yu xion''意思是``Arka被紫苑写.''


\emoji{x_loki} 
OK,Arka不用加``be,''我也费不上记动词的过去分词.只要加``yu'',然后被主宾关系倒过来就行了.

我逐渐地能写复杂的句子了呢.

但是我还不能翻译``这是一个苹果.''
Lein老师,你下次会教我Arka的系动词吧(\FiveStar ω\FiveStar)


%``([a-z]+)''
%``$1''



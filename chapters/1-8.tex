\chapter[祈使句]{祈使句}
%\chapter[短标题显示在页面]{长标题显示在目录}

 
\emoji{l_asex_kal} 
我们上次学了Arka的疑问句和否定句
紫苑,你把作业做的咋样了?


\emoji{x_pels} 
``它是猫吗?''; tu et ket mia?

``它不是猫,它是狗''; tu de ket. tu et oma.

``紫苑不写Arka吗?''; xion en axt arka mia?

--对吧?


\emoji{l_ket} 
OK,好样的!

最后一个句子是疑问句和否定句的组合,看起来你也做得不错呢.


\emoji{x_fron} 
副词``en,''后面接上动词表示否定.在英语里,类似的词``not,''出现在动词后面.

Arka里形容词在动词后面,但是``en''却在动词前面.这有点奇怪.

还有其他在动词前面的词吗?


\emoji{l_asex_kal} 
有,我接下来教你Arka的祈使句.

你要表示祈使,就在动词前面放``re''.``re lef''意思是``跑!''

我给你把出现在动词前面的词列成表
\begin{table}
    \begin{tabular}{|c|c|c|} % {l|c|r}<-- Alignments: 1st column left, 2nd middle and 3rd right, with vertical lines in between
      \hline
	  \textbf{词汇} & \textbf{简要翻译} & \textbf{意义}\\
      \hline
      re&  做&  命令\\\hline
  mir&  请做&  要求\\\hline
  den&  不准&  禁止\\\hline
  fon&  请不要&  劝阻\\\hline
    \end{tabular}
\end{table}

\emoji{x_lo} 
``请不要写''是``fon axt.''

你怎么说``请不要走''或者``请过来''?


\emoji{l_deyu} 
``fon ke''和``mir luna.''

``ke''是``走,''然后``luna''意思是``来.''


\emoji{x_asex} 
明白了.

这意味着祈使句可以向英语和汉语一样省略主语.

``you come here''也可以是``come here.''


\emoji{l_sena} 
这个省略不必强求.

``ti mir luna''就像``你过来一下.''


\emoji{x_tisse} 
在Arka里``You''是``ti''吗?

说到``你,'' 我还没有在Arka里学到``我'',``他''之类的代词呢.


\emoji{l_iks} 
唔...

\emoji{x_knoos} 
嗯?


\emoji{l_hik} 
其实我一直在......故意避免它们......


\emoji{x_knoos} 
为啥?


\emoji{l_reiv} 
因为真的很烦......(´·ω·`)





%``([a-z]+)''
%``$1''



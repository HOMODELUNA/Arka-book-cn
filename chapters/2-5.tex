\chapter{第一格(后)}

\emoji{x_asex}

好,到第一格的后半了.然后是"xom non fit est lex palue myu(xante"。

non是"我",fit是"给".理论编还记得住。

non在意义上应该是主语,不过,为什么主语前面先来了一个xom?

xom,xom......

----

[格词][文首纯词]"所以"、较弱的因果关系

19:较弱的son

----

\emoji{x_knoos} 

弱化版的son啊,那son又是什么?

----

[格词][文首纯词]既然、因此、中等結果意味

古:从so派生

[语法]

"因为"的对应。表示结果。


----


\emoji{x_lo} 
"既然"......啊。有"じゃあ"一样的感觉啊。

[文首纯词]这样标注,就像日语的接续词一样呢.

啊,这样啊。所以放在主语non前面。

xom non fit就是"那么我给你"的意思吧。

不过,给的东西......是哪个?


\emoji{l_hik} 

(我、放置play......(T\_{}T)


\emoji{x_loki} 
est是名字吗。给猫一个名字,哦.

然后是lex palue......?

呐,Alia,查lex出现了两个东西.

----

lex

[格词]给的对象,和yul值同一格。
%~と

[反意词]xel

15:恣意

lex(2)

[语言]l的文字

14:制:知識

[语法]

第20个幻字。最后的辅音字母。



----

\emoji{a_sm} 

既然lex(2)是字母的名字,这就是上面的那个。

yul指的是目的语哦。在这里指est(名字)就是目的语。

既然写了X lex Y,就有X = Y的意思。


\emoji{x_tisse} 
总之、est lex palue就理解成est = palue吗。

换句话说,"名字=palue"...啊,猫给猫起的名字就叫"palue"啊。


\emoji{l_ket} 

乒乓!\footnote{译注:奇奇怪怪的拟声词}

并且palue是"向阳处"的意思.暖洋洋的喵。

最后的myu啊――

----

[感動词][文末纯词]myu?,myun

19:新生词从古myup逆成
\footnote{日语词的形成法之一,
本来不是词尾的部分,由于形式上类似而被看作词尾,从而形成新词的现象。
由「たそがれ」形成「たそがる」,由double形成「ダブる」之类。}。

[语法]

猫风的文末纯词。表示心里愉悦平和


----

\emoji{x_pil} 

这也是文末纯词吗。看起来文末纯词可以在句子结尾表达说话者的心理呢。

sete是确认、myu是平和。

温和......我,我方了.


\emoji{a_niit} 

对于我们Arbazard人,语言是用来------

1:组织道理、

2:分析感情

――的工具哦。在内观和主张上都有特化.


\emoji{l_rana} 

这样一说,能细致地分析整理自己的感情,在Arbazard长大真是太好了。

连表示这些的专门形容词都有。

----

naxiris

[形容词][积极]心思纤细,可以感到机微.

[反意词]alnaxiris

20:na/ixirius(心之镜)

[语法]

用多种样式准确地表达自己的感情.
相对地,像对不论什么东西只是说"烦",就是alnaxiris。
naxiris有知性和风流的形象,具有积极色彩。
繊細而易伤,神经质等意味则不用这个词.在"知道细微的变化"这个意义上使用。
alnaxiris则有贬义。


----

\emoji{x_sena} 
这种形容词都有!?外语的学习变得有意思了啊。

总之、"xom non fit est lex palue myu (xante"的翻译就是"那么我就叫你洋洋\footnote{原文为ひだまり,照着起了类似的名字}吧♪"

呼------、总觉得漫画的第一格就累死了\~{}。


\emoji{l_niit} 
不论谁一开始都是这样哦。

不过从我看来,你学习的其实很惊人呢.

到此就休息吧,辛苦了.












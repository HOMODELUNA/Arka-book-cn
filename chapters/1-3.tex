\chapter[语序]{语序}
%\chapter[短标题显示在页面]{长标题显示在目录}

% {\small\textit{
% \quad Sors de l'enfance, ami réveille toi!\newline
% \quad \quad —Jean Jacques Rousseau.}
% \footnote{Ekster el infaneco,amiko veku!}
% \\}


\emoji{l_sena_deyu}
今天我们来学习Arka的语序吧!

汉语里,像``黑猫''这样的词,形容词在名词前面,但是在Arka里,你应该说``猫-黑.''


\emoji{x_loki}
你说...像法语那样? ``黑猫''在法语里叫做``chat noir''.形容词``noir''在名词``cat''后面.

在Arka里,就说``ket (猫) ver (黑)'',对吧?

明白了,形容词在名词后面.那么,你们是如何造句子的呢?


\emoji{l_niit}
就像汉语那样.

举个栗子,说``紫苑看到Lein,''你就像``紫苑'' + ``看'' + ``Lein''这样拼起来.你知道的,``主语'' + ``谓语'' + ``宾语.''

``看''在Arka里叫``in'',所以你要说``紫苑看到Lein,''就说:``xion in lein.''


\emoji{x_pil}
这就叫SVO.
在这一点上,Arka和汉语,英语都很像呢.


\emoji{l_deyu}
那么问题来了.
你怎么表达``大狗看小猫''呢?

单词: 大 = kai, 小 = lis, 狗 = oma, 猫 = ket,看 = in


\emoji{x_lo}
形容词在名词后面,so``大狗''就是``oma kai'',``小猫''就是``ket lis.''

语序是SVO,那么...

``Oma kai in ket lis,''这样?


\emoji{l_uni}
好!

嘿呀,你会用Arka造句子了.(v\^{}-°)

OK今天结束之前,我们再来看一个问题.

``给''在Arka里是``fit'',苹果是``miik''.So,``Lein给紫苑一个苹果''应该怎么表达?


\emoji{x_fron}
Arka是主-动-宾,就是``lein fit miik.''

慢着,``给紫苑''怎么说?


\emoji{l_niit}
我们用``a''指示目的方.
``A xion''就是``给紫苑''.合起来说就行了.


\emoji{x_tisse}
``A xion''就像英语短语一样,``to Shion''.``A''就像个介词.
所以答案是``lein fit miik a xion,''吗?
这句子还挺长的.我一开始对Arka一无所知,但现在我已经能说出一个句子了.这种不可思议的感觉...
Lein你怎么表达``看了''或者``给了''?


\emoji{l_sena}
我下次就教你Arka的时态. Doova ([doːva] (下次见)) !

%"([a-z]+)"
%``$1''



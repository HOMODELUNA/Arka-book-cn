\chapter[Arka的所有代词]{Arka的所有代词}
%\chapter[短标题显示在页面]{长标题显示在目录}
Arka的\hypertarget{appendix-pronouns}{位相}受到「个性位相、方言位相、环境位相、关系位相」四种要素约束,下面是10种
\footnote{
不止10种,注意下面的表,英语版的表没有[xia]一行,而是把[milia]一行当做[女-可爱]的代词序列.并且英语版没有方言,环境,关系等位相的描述,这里以日语版为准.
下面注明每种性格的时候,也没有提到[xia].
}
位相的表.\footnote{英语表中无相关译例,译者尝试用汉语重建译例,
	然而汉语的代词常用的是在太少,即使做出来了,很多都会有重复,参考作用不会太大,所以没有做.大家看日语版译例,可知一二
	}
\section{个性位相}

\begin{table}[H]
	\centering
		 
\resizebox{\textwidth}{!}{\begin{tabular}{|c|c|c|c|c|c|c|c|c|c|c|c|c|c|c|} % {l|c|r}<-- Alignments: 1st column left, 2nd middle and 3rd right, with vertical lines in between
		 \hline
	 性别&  性格&  位相名&  我&  我的&  我们&  我们的&  你&  你的&  你们&  你们的&  确认&  不确认&  传达& 日语译例
	 
	 \\\hline
	 男&  常规&  seet&  an&  ant&  ans&  antes&  ti&  tiil&  tiis&  tiiles&  kok&  sei&  tisee&私\\\hline
	 男&  强壮&  arden&  der&  dent&  orfen&  orfant&  dis&  din&  sedis&  sedin&  kaak&  daana&  dizmal&俺\\\hline
	 男&  粗鲁&  alben&  dai&  daid&  cuud&  cuudis&  baz&  bazel&  bcand&  bcendi&  daxxa&  zor&  laga&オレ\\\hline
	 男&  智性&  yuul&  ami&  amit&  sean&  seant&  tol&  tolte&  flent&  flandol&  ranxel&  fixet&  flenzel&僕\\\hline
	 男&  草食&  pikko&  men&  menal&  liia&  liian&  sala&  salen&  setyu&  setut&  xatta&  sivan&  ramma&ぼく\\\hline
	 女&   强势&   amma&   ema&   emil&   polte&   polton&   twa&   twal&   klans&   klansen&   soona&   yundi&   weeze&   アタシ\\\hline
	 女&   明朗&   mayu&   noel&   notte&   xenon&   xenoan&   xian&   xiant&   telul&   telet&   sanna&   enxe&   xiima&   あたし\\\hline
	 女&   胜气&   gano&   gan&   ganet&   kalas&   kaldes&   beg&   beget&   walet&   wolden&   yatte&   indi&   belze&   ウチ\\\hline
	 女&   绮丽&   yunk&   yuna&   yunol&   kolet&   ekol&   moe&   moen&   felie&   felial&   axem&   tuuyu&   linoa&   私\\\hline
	 女&   中立&   milia&   non&   noan&   lena&   lenan&   tyu&   tuan&   lilis&   lilin&   sete&   eyo&   tisse&   わたし\\\hline
	 女&   可爱&   xia&   xia&   txia&   xiif&   xifet&   myu&   myuan&   milis&   milin&   tanya&   poyo&   yuwa&   わたし\\\hline
	 中性&    &solferj&  noa&  noett&  lenoa&  lenoan&  tiam&  tiamet&  letiam&  letiant&  kotta&  ansem&  sante&私\\\hline
	 精灵&   & kotan&  nana&  nanatu&  nanasse&  nanassetu&  mina&  minatu&  minasse&  minassetu&  nano&  milei&  sian&ワタシ\\\hline
		\end{tabular}}
\end{table}
以性别和性格为基础分解,有12种位相.

男性有seet, arden, alben, yuul, pikko,默认情况是seet.

女性有amma, mayu, gano, yunk, milia。女性没有默认情况,根据性格决定位相,看心情有时候也用其他位相.
在公共场合发言时女性也可以用seet.个人对话时女性往往用yuul.
总之女性有很多选择.这一边女性既可以穿裤子也可以穿裙子,女性专用车厢和普通车厢都可以乘坐,像日本一样。

中性代词是Yult,Lyuu这样中性感的人物使用的.所谓"草食","男の娘"
\footnote{我很好奇他们是不是真的愿意明显地表示出来.也许这种勇气,和容许这种勇气的氛围,只在Kaldia存在吧}之人也适用。
相反,不像女性的中性女性也会使用,而不会表现"胜于男性"的气场.要不然就不用amma, mayu, gano,而直接用男性的代词了。

精霊代词是那些无生命者所用的.像付丧神或者凭依的思念体.

以下是大致的10类代词的解释.这个标准保证不论什么样的人都有合适的选择。傲娇之类易怒的女生会用mayu.
在孩子眼里长辈很多,于是使用pikko的场合也很多.同样的幼女会更多用yunk, milia.

seet:男性默认

arden:肉食男子、体育会

alben:粗野

yuul:智性

pikko:草食男子同样也做男性敬体

amma:大姐大,豪快,胜过男性

mayu:元气,明朗

gano:剩气,刚强

yunk:华丽系

milia:可爱系

\subsection{敬体}

男性用pikko与长辈讲话,女性则参考下表.
\begin{table}[H]
		\begin{tabular}{|c|c|c|c|c|c|c|c|c|c|c|} % {l|c|r}<-- Alignments: 1st column left, 2nd middle and 3rd right, with vertical lines in between
			\hline
			我&     我的&       我们&     我们的&     你&     你的&  你们&     你们的&     确认&  不确认&  传达\\\hline
			meid&   meiden&     xelis&  xelien&  halka&  halkan&  isfel&  isfelia&  malia&  nyannya&  yuulia\\\hline
	\end{tabular}
\end{table}

\subsection{第三人称}


第三人称只有mayu, yunk, milia是特别的,用法中特别的也只占一小部分.
\begin{table}[H]
		\begin{tabular}{|c|c|c|c|c|c|c|c|c|} % {l|c|r}<-- Alignments: 1st column left, 2nd middle and 3rd right, with vertical lines in between
		\hline
		seet&  luus&  luutes&  laas&  laates&  tuus&  tuules&  lees&  leetes\\\hline
		 mayu&  \textcolor{red}{luan}&  \textcolor{red}{luant}&  \textcolor{red}{lain}&  \textcolor{red}{laint}&  tuus&  tuules&  lees&  leetes\\\hline
		 yunk&  \textcolor{red}{luan}&  \textcolor{red}{luant}&  \textcolor{red}{lain}&  \textcolor{red}{laint}&  \textcolor{red}{tutu}&  \textcolor{red}{tutul}&  \textcolor{red}{lele}&  \textcolor{red}{leent}\\\hline
		 milia&   \textcolor{red}{luan}&  \textcolor{red}{luant}&  \textcolor{red}{lain}&  \textcolor{red}{laint}&  \textcolor{red}{tutu}&  \textcolor{red}{tutul}&  \textcolor{red}{lele}&  \textcolor{red}{leent}\\\hline

	\end{tabular}
\end{table}

\subsection{千金位相}

被称为大小姐的阶层在yunk, milia, mayu的任意一个位相像前面加上an.
例如anyuna.敬体在使用meid系列时也会在前面加上an,像anmeid.
这是装腔作势的贵族和有钱人的说法,虽然在现实中不太常见,但也不是零。动画和漫画在塑造大小姐角色时也会用到.
至于听起来像不像装腔作势的讨厌大小姐,取决于本人的性格和对方的接受方式。是讨厌,或是装腔作势,或是真正的大家闺秀,视情况而定。

anyuna yuus fan anmoe luna ankolet anlinoa(你我也许会相处甚欢
\footnote{英语:You may join us.日语:あなた、わたくしたちに混ぜてさしあげてもよくてよ.真的,我翻不准.
}
).

一直说an的做法被嘲为anankuom(anan语气),这么做会让你看起来像个马鹿.

再者,第三人称也加这个an,像anluan, antutu之类。

\subsection{位相的使用方法}


根据关系和语境切换位相是很常见的.例如对晚辈an使用der,对长辈则使用men.这和日语里不要对长辈使用"俺"是一个道理.
不过也有真的不换位相的人.就像有对长辈说"俺"之类粗话的人一样.

第一人称用ami而第二人称用dis也可以,这被叫做位相的交叉.
"我(僕)"是ami,对方根据与自己的关系,有时会变成"你(tol)",有时会变成"你(dis)\footnote{对应的日语分别是「君」和「お前」}"。
这和日语里用"僕"自称,但对别人称"君","あなた""お前"是一样的.
不过,自称am而对对方称tol是默认情况.同样的,自称der,称对方dis也是默认的.

\subsection{单词位相}

女性一般用xen(饮)代替kui(食べる).这时根据位相的不同,遣词也会变化.
举以下几个例子.这些单词在词典中多带有[rente],[郑重]等标签。

动词

kui→xen
ku→rens
mok→xidia
skin→fians

名词

xia→felver

形容词

vein→fein

副词

lax→lan
ris→rin
van→fan

格词

fin→fien

\section{方言位相}

在Arbazard同一片大地上Arsia使用西方方言,Kadesh使用南方方言,Arbazard使用北方方言.
每一处的语言都无疑是Arka,但语言学上将其按方言位相区分.


\section{环境位相}

根据身份和立场,语言也有不同.
同样的含义,由教师说出和由母亲说出就不一样.
公共场合就使用正规的文法,减少剧中的省略.同样的遣词也偏向于seet系.

\section{关系位相}
关系位相由彼此的环境相互决定.
关键是说话人的环境不会改变。

Lidia→Yult:"leen yultan, lax piyo nono?(呐,Yult,想要我们的小鸡雏?)
Lidia→Seren:"nee seren sou, tyu lax mive noan?"(啊,Seren君,要小鸡雏吗?
\footnote{原文分别是"ねーぇ、ユルトちゃん、あたちのぴよちゃんほしい?","あ、セレン君、ひよこ要る?"
}?)

%``([a-z]+)''
%``$1''
